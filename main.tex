\documentclass[11pt,landscape,a4paper]{article}
\usepackage[utf8]{inputenc}
\usepackage[T1]{fontenc}
%\usepackage[LY1,T1]{fontenc}
%\usepackage{frutigernext}
%\usepackage[lf,minionint]{MinionPro}
\usepackage{tikz}
\usetikzlibrary{shapes,positioning,arrows,fit,calc,graphs,graphs.standard}
\usepackage[nosf]{kpfonts}
\usepackage[t1]{sourcesanspro}
\usepackage{multicol}
\usepackage{wrapfig}
\usepackage[top=5mm,bottom=5mm,left=5mm,right=5mm]{geometry}
\usepackage[framemethod=tikz]{mdframed}
\usepackage{microtype}
\usepackage{pdfpages}
\usepackage[shortlabels]{enumitem}
\usepackage{flushend}
\usepackage{tcolorbox}

\raggedend 

\newif\iflong
\longtrue % \longfalse for short version and \longtrue for long version
\newcommand{\inLongVersion}[1]{\iflong #1\fi}
\newcommand{\HEADER}[1]{\begin{tcolorbox}
    \centering
    #1
\end{tcolorbox}}

\setlist{nosep}
\setlist[itemize]{leftmargin=*}
\setlist[enumerate]{leftmargin=*}

\newcommand{\score}{\text{score}}
\newcommand{\encoder}{\text{encoder}}
\newcommand{\decoder}{\text{decoder}}
\newcommand{\E}{\mathbb{E}}
\newcommand{\Dist}{\mathcal{D}}
\newcommand{\normal}{\mathcal{N}}
\DeclareMathOperator*{\dprime}{\prime \prime}
\DeclareMathOperator{\cov}{\textbf{Cov}}
\DeclareMathOperator{\var}{\textbf{Var}}
\DeclareMathOperator{\argmin}{\textbf{argmin}}
\DeclareMathOperator{\argmax}{\textbf{argmax}}
\DeclareMathOperator{\sgn}{\textbf{sgn}}
\DeclareMathOperator{\dir}{\textbf{Dir}}
\DeclareMathOperator{\cat}{\textbf{Cat}}
\DeclareMathOperator{\prob}{prob}
\DeclareMathOperator{\dom}{\textbf{dom}}
\DeclareMathOperator{\epi}{\textbf{epi}}
\DeclareMathOperator{\rint}{\textbf{rint}}
\DeclareMathOperator{\Prox}{Prox}
\DeclareMathOperator{\LMO}{LMO}
\DeclareMathOperator{\prox}{\textbf{prox}}






\let\bar\overline

\include{./def}

\begin{document}
%\footnotesize
\small
\begin{multicols*}{4}

\HEADER{Through the sheet, we would simplify $\|\cdot\|_2$ as $\|\cdot\|$ for vectors and spectral norm for matrices, if not explicitly say not so. For ease of reference, we include the reference to the lecture notes (the reference for the min-max section is to the slides) in the beginning of every referred property.}

\section{Introduction}

\textbf{A classical algorithm is evaluated on space and time complexity, but a learning algorithm is mainly evaluated on how well it explains the data.}

Definitions:
\begin{enumerate}
    \item A \textit{risk} or \textit{loss} is a function $l: \mathcal{H} \times \mathcal{X} \rightarrow \mathcal{R}$, representing how bad a given hypothesis $H \in \mathcal{H}$ is, on the given data $X$.
    \item The \emph{expected risk} is a function $l(H) := \E_\mathcal{X} (l(H, X))$, representing how bad a given hypothesis $H \in \mathcal{H}$ is, measured on the whole data distribution.
    \item An optimal hypothesis is a hypothesis $H^* := \argmin_{H\in\mathcal{H}} l(H)$.
    \item An \emph{empirical risk minimization (ERM)} hypothesis is a hypothesis $H^*(X):= \argmin_{H\in\mathcal{H}} l(H, X)$.
    \item The \emph{probably approximately correct (PAC)} theory characterizes the following: under which condition that given $\delta, \epsilon>0$, the ERM $\tilde{H}:=H^*(X)$ satisfies $l(\tilde{H}) \le \inf_{H\in \mathcal{H}} l(H) +\epsilon$ with probability at least $1-\delta$. The probability is over the sampling.
\end{enumerate}

\textbf{Theorem}: Suppose that for any $\delta, \epsilon>0$, there exists $n_{0} \in \mathbb{N}$ such that for $n \geq n_{0}$,
$
\sup _{H \in \mathcal{H}}\left|\ell_{n}(H)-\ell(H)\right| \leq \varepsilon
$
with probability at least $1-\delta$. Then, for $n \geq n_{0}$, an approximate empirical risk minimizer $\tilde{H}_{n}$ is PAC for expected risk minimization, meaning that it satisfies
$
\ell\left(\tilde{H}_{n}\right) \leq \inf _{H \in \mathcal{H}} \ell(H)+3 \varepsilon
$
with probability at least $1-\delta$.

Note: the condition $\sup _{H \in \mathcal{H}}\left|\ell_{n}(H)-\ell(H)\right| \leq \varepsilon$ is to ensure $\ell\left(\tilde{H}_{n}\right) \leq \ell_{n}\left(\tilde{H}_{n}\right)+\varepsilon$, which cannot be proved by the law of large numbers, because it cannot be applied to $\tilde{H}_{n}$ which is data dependent (thus dependent to the random variables involved in the law of large numbers).

\subsection{VC Theory}

Setting: binary classification, unknown distribution over the data source.

For $n$ given samples, each hypothesis $H \in \mathcal{H}$ induces a cut $H \cap \{x_1, \dots, x_n\}$. The set of all possible cuts by the hypothesis class on $n$ given samples is $\mathcal{H} \cap \{x_1, \dots, x_n\}:= \{H \cap \{x_1, \dots, x_n\} : H \in \mathcal{H} \}$. For any $n$ samples, the elements of this set is bounded by $2^n$, as each sample can be either 0 or 1. Further, define $\mathcal{H}(n) := \max_{x_1, \dots, x_n} |\mathcal{H} \cap \{x_1, \dots, x_n\}|$.

\textbf{Theorem (VC)}: Let $\mathcal{D}$ be a probability space, $\mathcal{H}$ a set of events, $n \in \mathbb{N}, \varepsilon>0$. Then
$
\sup _{H \in \mathcal{H}}\left|\prob_{n}(H)-\prob(H)\right|>\varepsilon
$
with probability at most
$
4 \mathcal{H}(2 n) \cdot \exp \left(-\varepsilon^{2} n / 8\right)
$.

Note that in this case, $\ell(H)=\prob(H)$, thus if $\mathcal{H}(n)$ is polynomially bounded, we have $\sup _{H \in \mathcal{H}}\left|\ell_{n}(H)-\ell(H)\right| \leq \varepsilon$ for sufficiently large $n$, which implies PAC property.

\section{Convex Functions and Optimization}

\textbf{Cauchy-Schwarz Inequality}: Let $u, v \in \mathbb{R}^d$, then $|\langle u,v\rangle| \le \|u\| \|v\|$ for any inner product space, where $\|\cdot\|$ is the induced norm of the inner product, defined as $\|u\| = \sqrt{\langle u,u\rangle}$.

\textbf{Spectral Nrom}: Let $A$ be a matrix of shape $m\times d$. The spectral norm of $A$ is $\|A\| = \max_{\|v\|_2=1} \|Av\|_2$.

\textbf{Definitions}:
\begin{enumerate}
    \item A set $S$ is convex if for any $x, y\in S$ and any $\lambda \in [0,1]$, $\lambda x + (1-\lambda) y \in S$.
    \item A function $f$ is convex if (1) its domain is convex, (2) the function value of any convex combination is below the convex combination of the function values, i.e., for all $x, y \in \dom(f)$ and all $\lambda \in [0,1]$, we have $f(\lambda x + (1-\lambda) y) \le \lambda f(x) + (1-\lambda) f(y)$.
    \item A function $f$ is strictly convex if the inequality above is strict.
    \item The epigraph of a function is the upper region characterized by the function, i.e., $\epi(f) = \{(x, y): x \in \dom(f) \land y \ge f(x) \}$.
\end{enumerate}

\textbf{General Properties}:
\begin{enumerate}
    \item (2.11) $f$ is convex iff $\epi(f)$ is a convex set. Proof by definition.
    \item Jensen's Inequality: for $\lambda_i \ge 0$, $\sum_i \lambda_i = 1$ and a convex function $f$, then $f(\sum_i \lambda_i x_i) \le \sum_i \lambda_i f(x_i)$ for any $x_i \in \dom(f)$. Proof by induction.
    \item (2.13) A convex function $f$ with $\dom(f)$ open and $\dom(f) \in \mathbb{R}^d$ is continuous. Proof: first show $f$ is bounded on any hypercube and has the extreme values on some corners. Then use sufficiently small cube to conclude continuity.
    \item \textbf{First-order characterization}: A differentiable function $f$ with open convex domain is convex iff for any $x,y\in \dom(f)$, we have $f(y) \ge f(x) + \nabla f(x)^\top (y-x)$. Proof: (1) One side: Taylor expansion for $f(x+t(y-x))$ on $t$ at $x$ and use $f(x+t(y-x))\le f(x) + t(f(y)-f(x))$; (2) The other side: introduce $z = \lambda x + (1-\lambda) y$, then apply the inequalities $f(x) \ge f(z) + \nabla f(z)^\top (x-z)$ and $f(y) \ge f(z) + \nabla f(z)^\top (y-z)$ and plug into $\lambda f(x) + (1-\lambda)f(y)$.
    \item \textbf{Monotonicity of the gradient}: A differentiable function $f$ with open convex domain is convex iff $(\nabla f(y) - \nabla f(x))^\top (y-x) \ge 0$. Proof: (1) One side: use first-order characterization alternatively for $x$ and $y$, then add these two inequalities. (2) The other side: define $h(t) = f(x+t(y-x))$, thus $h^\prime(t) = \nabla f(x+t(y-x))^\top (y-x)$. By monotonicity of the gradient, $h^\prime(t) \ge \nabla f(x)^\top (y-x)$ for $t \in (0,1)$. By mean-value theorem, there exists $c \in (0,1)$ such that $h(1)-h(0)=h^\prime(c) \ge \nabla f(x)^\top (y-x)$. Plugging in $h(1)=f(y)$ and $h(0)=f(x)$ suffices.
    \item \textbf{Second-order characterization}: A twice continuously differentiable function $f$ with open convex domain is convex iff $\nabla^2 f(x) \succeq 0$. Proof: (1) One side: use $h(t) = f(x+t(y-x))$ again. By monotonicity of the gradient, we can conclude $\frac{h^\prime(\delta) - h^\prime(0)}{\delta} \ge 0$, which implies $h^{\dprime}(0) \ge 0$. Note $h^{\dprime}(t) = (y-x)^\top \nabla^2 f(x+t(y-x)) (y-x)$, and thus $h^{\dprime} (0) \ge 0$ implies $\nabla^2 f(x) \succeq 0$. (2) The other side: mean-value theorem implies $h^\prime(1) - h^\prime(0) = h^{\dprime}(c)$ for some $c \in (0,1)$, which is $(\nabla f(y) - \nabla f(x))^\top (y-x) = (y-x)^\top \nabla^2 f(x+c(y-x)) (y-x) \ge 0$. This is the monotonicity of the gradient, thus $f$ is convex.
    \item (2.18) \textbf{Operations that perserve convexity}: For convex functions $\{f_i\}$, $f:=\max_i f_i$ and $f:=\sum_i  \lambda_i f_i$ for $\lambda_i \ge 0$ are convex on $\dom(f):=\cap_i \dom(f_i)$. For convex $f$ and affine $g$, $f \circ g$ is also convex.
\end{enumerate}

\textbf{Minimizer Properties}:
\begin{enumerate}
    \item (2.20) Every local minimum of a convex function is a global minimum. Proof by contradiction, moving a small step from the local minimum towards the global minimum violates convexity.
    \item (2.21 and 2.22) For a differentiable convex function with open domain, $\nabla f(x) = 0$ iff $x$ is a global minimum. Proof: (1) One side: first-order characterization. (2) The other side: contradiction by moving a small step in the gradient descent direction.
    \item (2.24) Positive definite Hessian implies strict convexity. Proof by Taylor expansion. The reverse is false, e.g., $f(x)=x^4$ is strictly convex, but its Hessian is not positive definite at 0.
    \item (2.25) A stricly convex function has at most one global minimum. Proof by contradiction.
    \item (2.27) For $f$ convex and differentiable over an open domain, $x^*$ is a minimizer over $\dom(f)$ iff $\nabla f(x^*)^\top (x-x^*) \ge 0$ for any $x \in \dom(f)$. Proof by first-order characterization.
\end{enumerate}

\subsection{Convex Programming}

\textbf{Definitions}:
\begin{enumerate}
    \item The standard form of a convex program is to minimize $f(x)$ s.t. $f_i(x) \le 0$ and $h_i(x) =0$, where $f$ and $f_i$ are convex and $h_i$ are affine. The feasible set is convex and the objective is convex.
    \item The Lagrangian of a program is $L(x, \lambda, \nu):= f(x) + \sum_i \lambda_i f_i(x) + \sum_i \nu_i h_i(x)$. The $\lambda_i$ and $\nu_i$ are called Lagrange multipliers.
    \item The Lagrange dual function is $g(\lambda_i, \nu_i) = \inf_{x\in\mathcal{D}} L(x, \lambda_i, \nu_i)$, where $\mathcal{D}$ is the feasible set.
    \item The Lagrange dual program is to maximize $g(\lambda_i, \nu_i)$ s.t. $\lambda_i \ge 0$.
\end{enumerate}

\textbf{Properties}:
\begin{enumerate}
    \item \textbf{Weak duality}: for any feasible $x$ and $\lambda_i \ge 0$, we have $g(\lambda_i, \nu_i) \le f(x)$, i.e., $\max_{\lambda_i \ge 0} g(\lambda_i, \nu_i) \le \inf_{x\in\mathcal{D}} f(x)$. Proof by noticing that for feasible $x$, $L(x, \lambda_i, \nu_i) \le f(x)$.
    \item \textbf{Strong duality (Slater's condition)} For convex program with a feasible $x$ such that the inequalities are stricly satisfied (Slater's point), then $\max_{\lambda_i \ge 0} g(\lambda_i, \nu_i) = \inf_{x\in\mathcal{D}} f(x)$. If this value is finite, then it is attained by a feasible solution of the dual program. Slater's condition is sufficient but not necessary for strong duality.
    \item \textbf{Strong duality (KKT condition)} Strong duality holds iff the followings hold for the solution (1) (complementary slackness) $\lambda_i f_i(x)=0$, (2) (vanishing gradient) $\nabla_x L(x, \lambda_i, \nu_i) =0$. (3) Primal feasibility and dual feasibility.
\end{enumerate}

\section{Gradient Descent}

\includegraphics[width=\linewidth]{imgs/GD.jpg}

Define $g_t = \nabla f(x_t)$. The gradient descent is $x_{t+1} = x_t - \gamma g_t$, where $\gamma$ is the step size. We assume $f$ is differentiable anywhere.

\textbf{Definitions}:
\begin{enumerate}
    \item A convex differentiable function $f$ is $L$-smooth if $f(y) \le f(x) + \nabla f(x)^\top (y-x) + \frac{L}{2}\|x-y\|^2$ for any $x, y$.
    \item A function is $\mu$-strongly convex if $f(y) \ge f(x) + \nabla f(x)^\top (y-x) + \frac{\mu}{2}\|x-y\|^2$ for any $x, y$.
\end{enumerate}

\textbf{Properties}:
\begin{enumerate}
    \item \textbf{Characterization of $L$-smoothness (equivalent)}: (3.3) $\frac{L}{2}x^\top x - f(x)$ is convex; (3.5) $\|\nabla f(x) - \nabla f(y)\| \le L \|x-y\|$ for any $x,y$;
    \item \textbf{Operations that perserve smoothness} (3.6): (i) Assume $f_i$ are $L_i$-smooth and $\lambda_i >0$, then $f:= \sum_{i} \lambda_i f_i$ is $\sum_i \lambda_i L_i$-smooth. Proof by (3.3). (ii) Assume $f$ is $L$-smooth and $g(x) = Ax+b$, then $f\circ g$ is $L\|A\|^2$-smooth.
    \item \textbf{Characterization of $\mu$-strongly convexity (equivalent)}: (3.11) $f(x) - \frac{\mu}{2}x^\top x$ is convex.
    \item \textbf{Bound of first-order changes}: Let $f$ be convex and $x^*$ be the global minimum, i.e. $\nabla f(x^*) = 0$. If $f$ is $\mu$-strongly convex, then $\nabla f(x)^\top (x - x^*) \ge \mu \|x-x^*\|^2$. If $f$ is $L$-smooth, then $\nabla f(x)^\top (x - x^*) \le L \|x-x^*\|^2$. Proof: write definition separately for $x^*$ on $x_t$ and $x_t$ on $x^*$, then add the two inequalities together.
\end{enumerate}

Analysis based on the first-order characterization:
\begin{enumerate}
    \item Vanilla analysis (no further assumption): by first-order characterization, we can bound $f(x_t)-f(x^*) \le g_t^\top (x_t - x^*)$. The algorithm says $g_t = (x_t - x_{t+1}) / \gamma$, thus $g_t^\top (x_t - x^*) = \frac{1}{\gamma} (x_t - x_{t+1})^\top (x_t - x^*)$. Since $2a^\top b = a^\top a + b^\top b - (a-b)^\top (a-b)$, we have $2(x_t - x_{t+1})^\top (x_t - x^*) = \|x_t - x_{t+1}\|^2 + \|x_t - x^*\|^2 - \|x_{t+1} -x^*\|^2$. Therefore, $\sum_{t=0}^{T-1}(f(x_t)-f(x^*)) \le \sum_{t=0}^{T-1} g_t^\top (x_t - x^*) \le \frac{\gamma}{2}\sum_{t=0}^{T-1}\|g_t\|^2 + \frac{1}{2\gamma} \|x_0-x^*\|^2$.  The problem is to bound the squared norm of gradients.
    \item Lipschitz convex functions: bounded gradients $\|\nabla f(x)\| \le B$ for any $x$. (3.1) Assume $\|x_0 - x^*\| \le R$.  Then the result of vanilla analysis says $\sum_{t=0}^{T-1}(f(x_t)-f(x^*)) \le \frac{\gamma}{2}B^2 T + \frac{1}{2\gamma}R^2$. Choose $\gamma=\frac{R}{B\sqrt{T}}$ yields $\frac{1}{T}\sum_{t=0}^{T-1}(f(x_t)-f(x^*)) \le \frac{RB}{\sqrt{T}}$. This means we need $T\ge R^2B^2/\epsilon^2$ to achieve $\min_t (f(x_t)-f(x^*)) \le \epsilon$. 
\end{enumerate}

Analysis based on control over the quadratic term:
\begin{enumerate}
    \item $L$-smooth functions (not requiring convexity): (3.7) with $\gamma:=1/L$, we have $f(x_{t+1}) \le f(x_t) - \frac{1}{2L}\|\nabla f(x_t)\|^2$. Proof: use smoothness definition and plug in $x_{t+1}-x_t = -\frac{1}{L}\nabla f(x_t)$.
    \item Convex $L$-smooth functions: (3.8) with $\gamma:=1/L$, we have $f(x_T) - f(x^*) \le \frac{L}{2T}\|x_0 - x^*\|^2$. This means we only need $T \ge \frac{R^2 L}{2\epsilon}$ to achieve error at most $\epsilon$. Proof: by (3.7), $\frac{1}{2L}\sum_{t=0}^{T-1} \|\nabla f(x_t)\|^2 \le f(x_0) - f(x^*)$. Plugging into vanilla analysis, we have $\sum_{t=1}^T (f(x_t) - f(x^*)) \le \frac{L}{2}\|x_0 - x^*\|^2$. By (3.7), $f(x_t) - f(x^*)$ is monotonically decreasing, thus $f(x_T) - f(x^*) \le \frac{L}{2T}\|x_0 - x^*\|^2$. 
    \item $\mu$-strongly convex and $L$-smooth functions: (3.14) GD with $\gamma = 1/L$ yields $\|x_{t+1}-x^*\|^2 \le (1-\frac{\mu}{L})\|x_t - x^*\|^2$ and $f(x_T) - f(x^*) \le \frac{L}{2}(1-\frac{\mu}{L})^T \|x_0 -x^*\|^2$. This means we need $T \ge \frac{L}{\mu}\log\left(\frac{R^2 L}{2\epsilon}\right)$ to achieve error at most $\epsilon$. Proof: (i) replacing the first-order characterization in the vanilla analysis by the condition of $\mu$-strongly convexity, we get $\|x_{t+1}-x^*\|^2 \le 2\gamma \left(f(x^*) - f(x_t)\right) + \gamma^2 \|\nabla f(x_t)\|^2 + (1-\mu\gamma)\|x_t - x^*\|^2$. By sufficient decrease of $L$-smooth functions, we have $f(x^*) - f(x_t) \le f(x_{t+1}) - f(x_t) \le -\frac{1}{2L}\|\nabla f(x_t)\|^2$. Combining these two gives the first result. (ii) By smoothness, $f(x_T) - f(x^*) \le \frac{L}{2}\|x_T - x^*\|^2 \le \frac{L}{2}(1-\frac{\mu}{L})^T \|x_0 -x^*\|^2$. 
\end{enumerate}

\textbf{Optimizing without knowing $L$ or $B$}: for $L$-smooth convex functions, we do not need to know $L$ to ensure $O(\frac{R^2 L}{\epsilon})$ steps. The idea is to guess $L$ and refine it gradually. The first guess is $L_0 := \frac{2\epsilon}{R^2}$. For each guess, we check if the sufficient decrease (3.7) holds. If (3.7) holds in the whole process $T = \frac{R^2 L_i}{2\epsilon}$, then $L_i$ is a successful guess, and we finish. If the guess is incorrect, we double $L_{i+1} = 2L_i$ and repeat. The final guess cannot exceed two times the correct value, and the number of iterations is bounded by $\sum_i 2^i$ until the last term exceeds the true $L$. This means the total number of iterations is bounded by two times the last iteration time, which is still $O(\frac{R^2 L}{\epsilon})$. A similar approach can be taken for Lipschitz convex functions to optimize without knowing $B$.

\textbf{Accelerated Gradient Descent}

The AGD is to do $y_{t+1}= x_t - \frac{1}{L}\nabla f(x_t)$, $z_{t+1} = z_t - \frac{t+1}{2L}\nabla f(x_t)$ and $x_{t+1} = \frac{t+1}{t+3} y_{t+1} + \frac{2}{t+3}z_{t+1}$. Initialized with $y_0=z_0 = x_0$.

(Thm 3.9) For $L$-smooth convex functions, AGD yields $f(y_T) - f(x^*) \le \frac{2L\|z_0 - x^*\|^2}{T(T+1)}$.

\section{Projected Gradient Descent}

PGD: do constrainted gradient descent. $y_{t+1} = x_t - \gamma \nabla f(x_t)$ then $x_{t+1} = \Pi_X(y_{t+1}) = \argmin_{X} \|x - y_{t+1}\|^2$.

\textbf{projection inequalities} (4.1): let $X$ be closed and convex, $x\in X$, then $(x - \Pi_X(y))^\top (y - \Pi_X(y)) \le 0$ and $\|x - \Pi_X(y)\|^2 + \|y - \Pi_X(y)\|^2 \le \|x - y\|^2$. Proof: $\Pi_X(y)$ minimizes $f(x) = \|x- y\|^2$ on $Z$, thus by optimality, $\nabla f(\Pi_X(y))^\top (x - \Pi_X(y)) = 2(\Pi_X(y) - y)^\top (x - \Pi_X(y)) \ge 0$ for any $x \in X$. The second follows from $2a^\top b = \|a\|^2 + \|b\|^2 - \|a - b\|^2$. Illustration:
    \begin{center}
        \includegraphics[width=.6\linewidth]{imgs/projection.jpg}
    \end{center}

\textbf{Analysis for PGD}: the same result but proof adapted using projection inequalities.
\begin{enumerate}
    \item Vanilla analysis: the same analysis for PGD gives $g_t^\top (x_t - x^*) = \frac{1}{2\gamma}\left(\gamma^2 \|g_t\|^2 + \|x_t - x^*\|^2 - \|y_{t+1}-x^*\|^2\right)$. By projection inequality, we have $\|x_{t+1} - x^*\|^2 \le \|y_{t+1} - x^*\|^2$. Thus, the result of vanilla analysis is the same.
    \item Lipschitz convex functions: (4.2) the same as GD since it only requires the result of vanilla analysis.
    \item $L$-smooth functions: (4.3) with $\gamma = 1/L$, we have $f(x_{t+1}) \le f(x_t) - \frac{1}{2L}\|\nabla f(x_t)\|^2 + \frac{L}{2}\|y_{t+1} - x_{t+1}\|^2$. In addition, $f(x_{t+1}) \le f(x_t)$. Proof: write the smoothness definition for $x_{t+1}$ for $x_t$. Replace $\nabla f(x_t)^\top (x_{t+1} - x_t)$ by $-L (y_{t+1} - x_t)^\top (x_{t+1} - x_t)$ and then apply $2a^\top b = \|a\|^2 + \|b\|^2 - \|a - b\|^2$ gives the first result. Since $\|y_{t+1} - x_{t+1}\| \le \|y_{t+1} - x_t\| = \gamma \|\nabla f(x_t)\|$, we have $f(x_{t+1}) \le f(x_t)$.
    \item Convex $L$-smooth functions: (4.4) the same result as GD. Proof: we use a tighter inequality for vanilla analysis. Instead of using $\|x_{t+1} - x^*\|^2 \le \|y_{t+1} - x^*\|^2$, we now use $\|x_{t+1} - x^*\|^2 + \|y_{t+1} - x_{t+1}\|^2 \le \|y_{t+1} - x^*\|^2$, so the vanilla analysis results in $\sum_{t=0}^{T-1} (f(x_t) - f(x^*)) \le \frac{1}{2L}\sum_{t=0}^{T-1} \|g_t\|^2 + \frac{L}{2}\|x_0 - x^*\|^2 - \frac{L}{2}\sum_{t=0}^{T-1} \|y_{t+1} - x_{t+1}\|^2$. Combining this with (4.3) gets the result.
    \item $\mu$-strongly convex and $L$-smooth functions: (4.5) PGD with $\gamma = 1/L$ yields $\|x_{t+1}-x^*\|^2 \le (1-\frac{\mu}{L})\|x_t - x^*\|^2$ and $f(x_T) - f(x^*) \le \frac{L}{2}(1-\frac{\mu}{L})^T \|x_0 -x^*\|^2 + (1-\frac{\mu}{L})^{T/2}\|\nabla f(x^*)\|\|x_0 - x^*\|$. This is still $O(\log(\frac{1}{\epsilon}))$ steps. Proof: $\mu$-strongly convexity strengthens the vanilla analysis to be $g_t^\top (x_t - x^*) \le \frac{1}{2\gamma}(\gamma^2 \|g_t\|^2 + \|x_t - x^*\|^2 - \|x_{t+1}-x^*\|^2 - \|y_{t+1} - x_{t+1}\|^2) - \frac{\mu}{2}\|x_t - x^*\|^2$. This makes the vanilla analysis to give $\|x_{t+1}-x^*\|^2 \le 2\gamma \left(f(x^*) - f(x_t)\right) + \gamma^2 \|\nabla f(x_t)\|^2 + (1-\mu\gamma)\|x_t - x^*\|^2 - \|y_{t+1} - x_{t+1}\|^2$. The extra $- \|y_{t+1} - x_{t+1}\|^2$ happens to compensate for the additional term from the $L$-smoothness, thus (i) follows. By smoothness, we have $f(x_T) - f(x^*) \le \|\nabla f(x^*)\| \|x_T - x^*\| + \frac{L}{2} \|x_T - x^*\|^2$, thus (ii) follows from (i).
\end{enumerate}

\textbf{Projecting into $L_1$ ball}: solve $\Pi_X(v) = \argmin_{\|x\|\le 1}\|x-v\|^2$. WLOG, assume $v_1 \ge v_2 \ge \dots v_d \ge 0$ and $\sum_i v_i > 1$. (4.11) We have $x_i^* = v_i - \theta_p$ for $i\le p$ and $x^*=0$ for $i>p$, where $\theta_p = \frac{1}{p}(\sum_{i=1}^p v_i -1)$ and $p=\max\{p \in [d]: v_p - \frac{1}{p}(\sum_{i=1}^p v_i - 1)>0\}$. This makes the projection $O(dlog d)$. Actually it can be improved to $O(d)$.

\section{PL Condition and Coordinate Descent}

\includegraphics[width=\linewidth]{imgs/CGD.jpg}

\subsection{PL Condition}

\textbf{Definitions}:
\begin{enumerate}
    \item \textbf{Polyak-Lojasiewicz condition}: we say $f$ satisfies PL condition if $\frac{1}{2}\|\nabla f(x)\|^2 \ge \mu (f(x) - f(x^*))$.
    \item \textbf{Strong convexity w.r.t. $\|\cdot\|_1$}: $f(y) \ge f(x) + \nabla f(x)^\top (y-x) + \frac{\mu}{2}\|y-x\|_1^2$.
\end{enumerate}

\textbf{Properties}:
\begin{enumerate}
    \item \textbf{PL condition is weaker than strong convexity}: (5.2) if $f$ is $\mu$-strongly convex, then $f$ satisfies PL condition for the same $\mu$. Proof: by strong convexity, $f(x^*) - f(x) \ge \|\nabla f(x)\| \|y-x\| + \frac{\mu}{2} \|y-x\|^2 \ge -\frac{1}{2\mu}\|\nabla f(x)\|^2$.
    \item \textbf{PL condition is strictly weaker than strong convexity}: $f(x_1, x_2) = x_1^2$ is not strongly convex, but satisfies PL condition.
    \item \textbf{Smooth functions satisfying PL condition can be solved by GD in $O(\log(\frac{1}{\epsilon}))$}: for such functions, we have $f(x_T) - f(x^*) \le (1-\frac{\mu}{L})^T (f(x_0) - f(x^*))$. Proof: by sufficient descrease of $L$-smoothness, we have $f(x_{t+1}) \le f(x_t) - \frac{1}{2L}\|\nabla f(x_t)\|^2$. By PL condition, this means $f(x_{t+1}) \le f(x_t) - \frac{\mu}{L} (f(x_t) - f(x^*))$, which implies $f(x_{t+1}) - f(x^*) \le (1-\frac{\mu}{L}) (f(x_t) - f(x^*))$.
    \item \textbf{Strong convexity w.r.t. $\|\cdot\|_1$ and $\|\cdot\|_2$}: if $f$ is $\mu$-strongly convex w.r.t. $\|\cdot\|_1$, then $f$ is also $\mu$-strongly convex w.r.t. $\|\cdot\|_2$; if $f$ is $\mu$-strongly convex w.r.t. $\|\cdot\|_2$, then $f$ is $\mu / d$-strongly convex w.r.t. $\|\cdot\|_1$. Proof: use $\|x\|_1 \ge \|x\|_2$ and $\|x\|_2 \ge \frac{1}{\sqrt{d}}\|x\|_1$.
    \item \textbf{Strong convexity w.r.t. $\|\cdot\|_1$ implies PL condition w.r.t. $\|\cdot\|_\infty$}: (5.9) if $f$ has strong convexity w.r.t. $\|\cdot\|_1$, then $\frac{1}{2}\|\nabla f(x)\|_\infty^2 \ge \mu (f(x) - f(x^*))$. Proof: simply use $\nabla f(x)^\top (y-x) \le \|\nabla f(x)\|_\infty \|y-x\|_1$ in (5.2) instead.
\end{enumerate}


\subsection{Coordinate Descent}

\textbf{Coordinate-wise smoothness}: $f$ is called coordinate-wise smooth with parameter $L = (L_1, \dots, L_d)$ if for every coordinate $i$ we have $f(x + \lambda e_i) \le f(x) + \lambda \nabla_i f(x) + \frac{L_i}{2} \lambda^2$.

\textbf{Coordinate Descent Algorithm}: for each iteration $t$, choose an active coordinate $i \in [d]$, then do $x_{t+1} = x_t - \gamma_i \nabla_i f(x_i) e_i$.

\textbf{Sufficient decrease}: (5.5) with $\gamma_i = 1 / L_i$, we have $f(x_{i+1}) \le f(x_i) - \frac{1}{2L_i} |\nabla_i f(x_t)|^2$. Proof: plugging the update step into the definition of coordinate-wise smoothness immediately gives the result.

\textbf{Analysis for coordinate-wise smooth functions satisfying PL condition}:
\begin{enumerate}
    \item \textbf{Randomized coordinate descent}: (5.6) choose coordinate uniformly at random and set $\gamma_i = 1/L$ where $L = \max_i L_i$, the randomized coordinate descent gets $\E(f(x_T) - f(x^*)) \le (1-\frac{\mu}{d L})^T (f(x_0) - f(x^*))$. This means randomized coordinate descent is the same good as GD, as the number of iterations is $d$ times higher, but each iteration is $d$ times cheaper. Proof: by sufficient decrease of coordinate descent, $\E (f(x_{t+1}) \mid x_t) \le f(x_t) - \frac{1}{2L} \sum_{i=1}^d \frac{1}{d} |\nabla_i f(x_t)|^2 = f(x_t) - \frac{1}{2dL} \|\nabla f(x_t)\|^2$. By PL condition, this means $\E (f(x_{t+1}) \mid x_t) \le f(x_t) - \frac{\mu}{dL}(f(x_t) - f(x^*))$, which implies $\E(f(x_{t+1}) - f(x^*) \mid x_t) \le (1-\frac{\mu}{dL})(f(x_t) - f(x^*))$. This means $\E(f(x_{t+1}) - f(x^*)) \le (1-\frac{\mu}{dL})\E(f(x_t) - f(x^*))$.
    \item \textbf{Importance sampling}: (5.7) choose coordinate $i$ with probability $\frac{L_i}{\sum_j L_j}$ and define $\bar{L} = \frac{1}{d}\sum_{i=1}^d L_i$, we have $\E(f(x_T) - f(x^*)) \le (1-\frac{\mu}{d\bar{L}})^T (f(x_0) - f(x^*))$. Note how randomized coordinate descent for $L$-smooth functions is a special case of this result. Proof: $\E (f(x_{t+1}) \mid x_t) \le f(x_t) - \frac{1}{2} \sum_{i=1}^d \frac{L_i}{\sum_j L_j} \frac{1}{L_i} |\nabla_i f(x_t)|^2 = f(x_t) - \frac{1}{2d\bar{L}}\sum_i |\nabla_i f(x_t)|^2$. The rest is the same to (5.6).
    \item \textbf{Steepest coordinate descent}: (5.8) choose coordinate $i = \argmax_i |\nabla_i f(x_t)|$, we have $f(x_T) - f(x^*) \le (1-\frac{\mu}{d L})^T (f(x_0) - f(x^*))$. Proof: simply remove the expectation and use maximum is greater than average.
    \item \textbf{Steepest descent for strong convexity w.r.t. $\|\cdot\|_1$}: (5.10) if $f$ is $\mu_1$-strongly convex w.r.t. $\|\cdot\|_1$, then with $\gamma_i = 1/L$, steepest descent gives $f(x_T) - f(x^*) \le (1- \frac{\mu_1}{L})^T (f(x_0) - f(x^*))$. Note that steepest descent has the same per iteration cost as GD, but the $\mu_1$ could be greater than the $L_2$ strong convexity parameter. Proof: by sufficient decrease and the update rule, $f(x_{t+1}) \le f(x_t) - \frac{1}{2L}\|\nabla f(x_t)\|_\infty^2$. By the PL condition w.r.t. $\|\cdot\|_\infty$, this means $f(x_{t+1}) \le f(x_t) - \frac{\mu_1}{L} (f(x_t) - f(x^*))$ and thus $f(x_{t+1}) - f(x^*) \le (1- \frac{\mu_1}{L})(f(x_t) - f(x^*))$.
    \item \textbf{Greedy coordinate descent}: choose some coordinate, then do $x_{t+1} = \argmin_\lambda f(x_t + \lambda e_i)$, i.e., make the largest step possible. For differentiable convex functions, as the update can only be better than before, it does not compromise the analysis. For non-differentiable case, however, it may get stuck in non-optimal points. (5.11) Assume $f(x) = g(x) + h(x)$, where $h(x) = \sum_i h_i (x_i)$, $g$ is convex and differentiable and $h_i$ is convex, then if the greedy descent converges, it converges to the global minimum of $f$. Such $h$ is called \emph{separable}, which includes $L_1$ norm and squared $L_2$ norm. This means LASSO and ridge objectives are concrete cases. Proof: $f$ is the sum of convex functions, thus convex. Let $x$ be the converged point and $y$ be a near point. Thus, $\nabla_i g(x) (y_i - x_i) + h_i(y_i) - h_i(x_i) \ge 0$, since the algorithm converges. By first-order characterization, we have $f(y) - f(x) \ge \nabla g(x)^\top (y-x) + \sum_i (h_i(y_i) - h_i(x_i)) \ge 0$.
\end{enumerate}

\section{Subgradient Methods}

\textbf{Separating plane of two convex sets}: let $S$ and $T$ be two nonempty convex sets. A hyperplane $a^\top x = b$ is said to separate $S$ and $T$ if $S \cup T \not\subset H$, $S \subset H^-=\{x: a^\top x \le b\}$ and $T \subset H^+ = \{x: a^\top x \ge b\}$. If further $S \subset H^{--}=\{x: a^\top x < b\}$ and $T \subset H^{++} = \{x: a^\top x > b\}$, then $H$ is said to strictly separate $S$ and $T$.

\textbf{Hyperplane separation theorem}: for two non-empty convex sets $S$ and $T$, they can be separated by a hyperplane iff $\rint(S) \cap \rint(T) = \emptyset$.

\textbf{Definitions}:
\begin{enumerate}
    \item \textbf{Subgradient}: let $f$ be a convex function, then a vector $g$ is a subgradient of $f$ at $x$ if $f(y) \ge f(x) + g^\top (y-x)$.
    \item \textbf{Subdifferential}: the set of all subgradient at $x$ is called the subdifferential of $f$ at $x$, denoted as $\partial f$.
    \item \textbf{Directional derivative}: the directional derivative of $f$ at $x$ along $d$ is $f^\prime(x;d) = \lim_{\delta \rightarrow 0^+} \frac{f(x+\delta d) - f(x)}{\delta}$.
\end{enumerate}

\textbf{Properties}:
\begin{enumerate}
    \item \textbf{subgradient = gradient, when convex and differentiable}: (6.2) if $f$ is convex and differentiable at $x$, then $\partial f=\{\nabla f(x)\}$; (6.3) if $f$ is only differentiable but not convex, then $\partial f \subseteq \{\nabla f(x)\}$. Proof: clearly $\{\nabla f(x)\} \subseteq \partial f$ given $f$ convex; assume $\partial f \ne \emptyset$, let $y = x+\epsilon d$ for small $\epsilon$, then $\frac{f(y) - f(x)}{\epsilon} \ge g^\top d$, which means $\nabla f(x)^\top d \ge g^\top d$ for any $d$. This implies $g = \nabla f(x)$.
    \item \textbf{Bounded subgradient for convex functions = Lipschitz}: (6.5) if $f$ is convex and $\dom(f)$ is open, then $\|g\| \le B$ for all $x$ and $g \in \partial f(x)$ is equivalent to $|f(x)-f(y)| \le B \|x-y\|$ for all $x,y$. Proof: (i) one side: let $y=x+\epsilon g$ for some $\epsilon >0$. Then $f(y)-f(x) \ge \epsilon \|g\|^2$. By $B$-Lipschitz, $f(y) -f(x) \le B\|y-x\| = \epsilon B \|g\|$. These two imply $\|g\|\le B$. (ii) the other side: for any $x,y$ and $g \in \partial f(x)$, $f(x) - f(y) \le g^\top (x-y) \le \|g\|\|x-y\| \le B\|x-y\|$ and $f(y) - f(x) \ge g^\top (y-x) \ge -B \|y-x\|$.
    \item \textbf{Convexity means subgradient almost everywhere}: (6.10) let $f$ be convex and $x \in \rint(\dom(f))$, then $\partial f(x)$ is non-empty and bounded. Proof: (i) non-emptiness: w.l.o.g, assume $\dom(f)$ is full-dimensional and $x \in \text{int}(\dom(f))$. Since $\epi(f)$ is convex, by hyperplane separation theorem, $\exists s, \beta$, s.t. $s^\top y + \beta t \ge s^\top x + \beta f(x)$ for any $(y,t) \in \epi(f)$. Since $t$ can be arbitratily large, we have $\beta \ge 0$. If $\beta = 0$, then $s^\top (y-x) \ge 0$ for any $y$, which means $s=0$, a contradiction. Thus, $\beta > 0$. Setting $g = -\beta^{-1} s$, we get $f(y) \ge f(x) + g^\top (y-x)$. (ii) boundness: suppose $\exists g_k \in \partial f(x)$ s.t. $\|g_k\| \rightarrow +\infty$. Since $x\in\text{int}(\dom(f))$, $\exists \delta>0$, s.t. $B(x, \delta) \subseteq \dom(f)$. Therefore, $y_k := x+\delta \frac{g_k}{\|g_k\|} \in \dom(f)$. By definition, $f(y_k) \ge f(x) + g_k^\top (y_k - x) = f(x) + \delta \|g_k\| \rightarrow +\infty$, a contradition.
    \item \textbf{Subgradient everywhere means convexity}: if $\dom(f)$ is convex and for any $x \in \dom(f)$, $\partial f(x)$ is non-empty, then $f$ is convex. Proof: for any $x, y \in \dom(f)$ and $\lambda \in (0,1)$, let $z = \lambda x + (1-\lambda) y$ and $g \in \partial f(z)$. Then $f(x) \ge f(z) + g^\top (x-z)$ and $f(y) \ge f(z) + g^\top (y-z)$. Adding them leads to definition of convexity.
    \item \textbf{$\text{dist}(0, \partial f(x))$ decides optimality}: if $0 \in \partial f(x)$, then $x$ is a global minimum. Proof: by definition.
    \item \textbf{Subgradient and directional derivative}: (6.13) let $f$ be convex and $x \in \text{int}(\dom(f)),$ then $f^\prime(x ;d) = \max_{g \in \partial f(x)} g^\top d$. Proof: by the definition of subgradient. we have $f(x+\delta d) - f(x) \ge \delta g^\top d$, which implies $f^\prime(x ; d) \ge g^\top d$ for any $g \in \partial f(x)$ and thus $f^\prime(x ; d) \ge \max_{g \in \partial f(x)} g^\top d$. To conclude the other side, consider $C_1 = \{(y,t): f(y)<t\}$ and $C_2 = \{(y,t): y=x+\alpha d, t = f(x)+\alpha f^\prime(x;d), \alpha \ge 0\}$. Clearly they are convex and non-empty. If $C_1 \cap C_2 = \emptyset$, then by hyperplane separation theorem, $\exists g_0, \beta$, s.t. $g_0^\top (x+\alpha d) + \beta (f(x) + \alpha f^\prime(x;d)) \le g_0^\top y + \beta t$ for any $\alpha \ge 0$ and any $t > f(y)$. Similar to the proof of (6.10), we can show $\beta > 0$. Let $\tilde{g}=\beta^{-1} g_0$, we have $\tilde{g}^\top (x+\alpha d) + f(x) + \alpha f^\prime(x;d) \le \tilde{g}^\top y + f(y)$ . Set $\alpha = 0$, we have $\tilde{g}x + f(x) \le \tilde{g}^\top y + f(y)$, which means $-\tilde{g} \in \partial f(x)$. Further, set $y=x$ and $\alpha=1$, we have $f^\prime(x;d) \le - \tilde{g}^\top d \le \max_{g \in \partial f(x)} g^\top d$.
\end{enumerate}

\textbf{Calculus of subgradient}:
\begin{enumerate}
    \item For $h(x) = \lambda f(x) + \mu g(x)$ where $\lambda, \mu \ge 0$ and $f,g$ both are convex, then $\partial h(x) = \lambda \partial f(x) + \mu \partial g(x)$.
    \item For $h(x) = f(Ax+b)$ where $f$ is convex, then $\partial h(x) = A^\top \partial f(Ax+b)$.
    \item For $h(x) = \sup_{\alpha \in A} f_\alpha(x)$ and each $f_\alpha(x)$ is convex, then $\partial h(x) \supseteq \text{conv}\{\partial f_\alpha(x): f_\alpha(x) = h(x)\}$.
    \item For $h(x) = F(f_1(x), \dots, f_m(x))$ where $F$ is non-decreasing and convex, then $\partial h(x) \supseteq \{\sum_{i=1}^m d_i \partial f_i(x): (d_1, \dots, d_m) \in \partial F(y_1, \dots, y_m)\}$.
\end{enumerate}

\subsection{Subgradient descent} 
Consider $f$ convex (possibly non-differentiable) on a closed and convex set $X$. The subgradient descent does $x_{t+1} = \Pi_X(x_t - \gamma_t g(x_t))$, where $g(x_t) \in \partial f(x_t)$. When $f$ is differentiable, this reduces to PGD. However, moving towards the negative direction of subgradient is not necessarily decreasing the objective. Therefore, we can only measure $\min_{t \le T} f(x_t) - f^*$ instead of $f(x_T)-f^*$.

\includegraphics[width=\linewidth]{imgs/subGD.jpg}

\textbf{Analysis for convex functions}:
\begin{enumerate}
    \item \textbf{General analysis}: (6.17) starting from $x_1$, we have $\min_{1\le t\le T} f(x_t) - f^* \le \frac{1}{2}(\sum_{t=1}^T \gamma_t)^{-1}(\|x_1 - x^*\|^2 + \sum_{t=1}^T \gamma_t^2\|g(x_t)\|^2)$ and $f(\hat{x}_T) - f^* \le \frac{1}{2}(\sum_{t=1}^T \gamma_t)^{-1}(\|x_1 - x^*\|^2 + \sum_{t=1}^T \gamma_t^2\|g(x_t)\|^2)$ for $\hat{x}_t=(\sum_{t=1}^T \gamma_t)^{-1}(\sum_{t=1}^T \gamma_t x_t)$. Proof: by definition, $\|x_{t+1} - x^*\|^2 = \|\Pi_X(x_t - \gamma_t g(x_t)) - x^*\|^2 \le \|x_t - \gamma_t g(x_t) - x^*\|^2 = \|x_t - x^*\|^2 - 2\gamma_t g(x_t)^\top (x_t - x^*) + \gamma_t^2\|g(x_t)\|^2$. By definition of subgradient, we have $f(x_t) - f^* \le g^\top (x_t - x^*)$. These two gives $\sum_{t=1}^T \gamma_t (f(x_t) - f^*) \le \frac{1}{2}(\|x_1 - x^*\|^2 - \|x_{T+1} - x^*\|^2 + \sum_{t=1}^T \gamma_t^2\|g(x_t)\|^2) \le \frac{1}{2}(\|x_1 - x^*\|^2 + \sum_{t=1}^T \gamma_t^2\|g(x_t)\|^2)$. In addition, we have $\min_{t} f(x_t) - f^* \le (\sum_{t=1}^T \gamma_t)^{-1} (\sum_t \gamma_t (f(x_t) - f^*))$, thus the first claim follows. For the second claim, use $\sum_t \gamma_t (f(x_t) - f^*) \ge (\sum_t \gamma_t) (f(\hat{x}_t) - f^*)$ by convexity.
    \item \textbf{Lipschitz functions}: assume $\|x_1 - x^*\| \le R$ and $\|g(x_t)\| \le B$, then $\min f(x_t) - f^* \le \frac{1}{2}(\sum_{t=1}^T \gamma_t)^{-1} (R^2 + \sum_t \gamma_t^2 B^2)$.
\end{enumerate}

\textbf{$O(1/\sqrt{t})$ convergence under different stepsizes for  convex Lipschitz functions}:
\begin{enumerate}
    \item $\gamma_t = \gamma$: $\epsilon_t = \frac{1}{2}(\frac{R^2}{T\gamma} + B^2\gamma) \rightarrow \frac{B^2}{2}\gamma$. Choose $\gamma_t = \frac{R}{B\sqrt{T}}$, we have $\epsilon_t \le \frac{RB}{\sqrt{T}}$.
    \item $\sum_t \gamma_t \rightarrow +\infty$ and $\gamma_t \rightarrow 0$: we have $\epsilon_t \rightarrow 0$. If set $\gamma_t = \frac{R}{B\sqrt{t}}$, we get $\epsilon_t = O(\frac{BR}{\sqrt{T}})$
    \item $\sum_t \gamma_t \rightarrow +\infty$ and $\sum_t \gamma_t^2 < +\infty$: $\epsilon_t \rightarrow 0$.
    \item Polyak stepsize $\gamma_t = \frac{f(x_t) - f^*}{\|g(x_t)\|^2}$: this makes the general analysis to give $\|x_{t+1} - x^*\|^2 \le \|x_t - x^*\|^2 - 2\gamma_t g(x_t)^\top (x_t - x^*) + \gamma_t^2 \|g(x_t)\|^2 \le \|x_t - x^*\|^2 - 2\gamma_t (f(x_t) - f^*) + \gamma_t^2 \|g(x_t)\|^2 = \|x_t - x^*\|^2 - \frac{(f(x_t) - f^*)^2}{\|g(x_t)\|^2} \le  \|x_t - x^*\|^2 - \frac{(f(x_t) - f^*)^2}{B^2}$, thus guarantees the decrease of $\|x_t - x^*\|$. This implies $\sum_t (f(x_t) - f^*)^2 \le R^2 B^2$, thus $\epsilon_t = O(\frac{RB}{\sqrt{T}})$.
\end{enumerate}

\textbf{$O(1/t)$ convergence for strongly convex Lipschitz functions}:
\begin{enumerate}
    \item (6.18) Let $f$ be $\mu$-strongly convex. With $\gamma_t = \frac{1}{\mu t}$, we have $\min f(x_t) - f^* \le \frac{B^2(\log(T)+1)}{2\mu T}$ and $f(\hat{x}_T) - f^* \le \frac{B^2(\log(T)+1)}{2\mu T}$ for $\hat{x}_T = \frac{1}{T}\sum_t x_t$. Proof: by strong convexity, $f(y) - f(x) + g(x)^\top (y-x) + \frac{\mu}{2}\|y-x\|^2$. Thus, general analysis gives $\|x_{t+1} - x^*\|^2 \le \|x_t-x^*\|^2 - 2\gamma_t (f(x_t) - f^* + \frac{\mu}{2}\|x_t - x^*\|^2) + \gamma_t^2 \|g(x_t)\|^2$, which implies an upper bound on $f(x_t) - f^*$. Following the same steps in the general analysis, we get the desired result.
    \item (6.19) Let $f$ be $\mu$-strongly convex. With $\gamma_t = \frac{2}{\mu (t+1)}$, we have $\min f(x_t) - f^* \le \frac{2B^2}{\mu (T+1)}$ and $f(\hat{x}_T) - f^* \le \frac{2B^2}{\mu (T+1)}$ for $\hat{x}_T = \sum_t \frac{2t}{T(T+1)} x_t$. Proof: the same analysis gives $t(f(x_t) - f^*) \le \frac{\mu t(t-1)}{4}\|x_t - x^*\|^2 - \frac{\mu t (t+1)}{4}\|x_{t+1}-x^*\|^2 + \frac{B^2}{\mu (t+1)}$. The result follows.
\end{enumerate}

\textbf{Subgradient descent is asymptotically optimal for first-order subgradient methods}: (Thm 6.20): for any $1\le t \le n$ and $x_1 \in \mathbb{R}^n$, there exists a $B$-Lipschitz continuous function $f$ and a convex set $X$ with diameter $R$, s.t. for any first-order method that generates $x_t \in x_1 + \text{span}(g(x_1), \dots, g(x_{t-1}))$, where $g(x_i) \in \partial f(x_i)$, we have $\min_{1\le s\le t}f(x_s) - f^* \ge \frac{BR}{4(1+\sqrt{t})}$. In addition, there exists a $B$-Lipschitz continuous function $f$ that is $\mu$-strongly convex, s.t. $\min_{1\le s\le t}f(x_s) - f^* \ge \frac{B^2}{8\mu t}$. Proof: W.l.o.g., assume $x_1=0$. Let $X = \{x:\|x\| \le R/2\}$ and $f(x) =  C\max_{1\le i\le t}x_i + \frac{\mu}{2}\|x\|^2$ for some $C>0, \mu >0$. The subgradient of $f$ is $\partial f(x) = \mu x + C \cdot \text{conv}\{e_i: i \text{ s.t. } x_i=\max_{1\le j \le t}x_j\}$. The optima of $f$ is $f^* = - \frac{C^2}{2\mu t}$. Consider $g(x)=Ce_i+\mu x$, where $i$ is the first coordinate that $x_i = \max_{1\le j \le t} x_j$. Since the algorithm runs for $t$ iterations, the last dimension cannot be updated. Therefore, $\min_{1\le s \le t}f(x_s) - f^* \ge \frac{C^2}{2\mu t}$. (1) Let $C = \frac{B\sqrt{t}}{1+\sqrt{t}}$ and $\mu = \frac{2B}{R(1+\sqrt{t})}$, we have $\max_{g \in \partial f(x)}\|g\| \le C + \mu \|x\|\le R$, and the first result follows. (2) Let $C=\frac{B}{2}$ and $\mu = \frac{B}{R}$, we have $\max_{g \in \partial f(x)}\|g\| \le C + \mu \|x\|\le R$, $f$ is $\mu$-strongly convex, and the second result follows.

\subsection{Mirror Descent}

Mirror descent only uses subgradients, and have the same aymptotic performance as subgradient descent. However, it may have better constants than subgradient descent. Subgradient descent is a special case of mirror descent.

\textbf{Definitions}:
\begin{enumerate}
    \item \textbf{Bregman Divergence}: let $w(x)$ be strictly convex and continuously differentiable on a close convex $X$. $V_w(x, y) = w(x) - w(y) - \nabla w(y)^T (x-y)$ is defined to be the Bregman divergence. This is asymmetric, thus not a valid distance. If $w$ is $\mu$-strongly convex, then $V_w(x, y)$ is $\mu$-strongly convex in $x$.
    \item \textbf{Prox-mapping}: given an input $x$ and vector $\xi$, the prox-mapping is defined as $\Prox_x(\xi) = \argmin_{u\in X}\{V_w(u, x) + \langle \xi, u \rangle\}$, where $w$ is $1$-strongly convex. 
    \item \textbf{Mirror Descent}: do $x_{t+1} = \Prox_{x_t}(\gamma_t g(x_t)) = \argmin_{x\in X}\{w(x)+\langle \gamma_t g(x_t)-\nabla w(x_t), x\rangle\}$. Recall that the subgradient descent is equivalent to $x_{t+1}=\argmin_{x \in X} \{\frac{1}{2}\|x-x_t\|^2 + \langle \gamma_t g(x_t), x\rangle\}$ and $V_w(x,y)=\frac{1}{2}\|x-y\|^2$ for $w(x) = \frac{1}{2}\|x\|^2$. Therefore, subgradient descent is a special case of mirror descent.
\end{enumerate}

\textbf{Properties}:
\begin{enumerate}
    \item \textbf{Three point identity}: (6.26) $V_w(x, z) = V_w(x, y) + V_w(y, z) - \langle \nabla w(z) - \nabla w(y), x-y \rangle$. Proof: by definition.
    \item \textbf{Generalized Pythagorean Theorem for Bregman Divergence}: (6.23) if $x^* = \argmin_{x \in C} V_w(x, x_0)$ for some convex set $C$, then for any $y \in C$, we have $V_w(y, x_0) \ge V_w(y, x^*) + V_w(x^*, x_0)$. Proof: use $\nabla V_w(x, x_0)^\top \mid_{x=x^*} (y-x^*) \ge 0 \Leftrightarrow (\nabla w(x^*) - \nabla w(x_0))^\top (y-x^*) \ge 0$ and the three point identity.
    \item \textbf{Convergence of mirror descent}: (6.28) let $f$ be convex, we have $\min_{1 \le t \le T}f(x_t) - f^* \le \frac{1}{\sum_{t=1}^T \gamma_t} (V_w(x^*, x_1) + \frac{1}{2}\sum_{t=1}^T \gamma_t^2 \|g(x_t)\|^2)$. Proof: since $x_{t+1} =  \argmin_{x\in X}\{w(x)+\langle \gamma_t g(x_t)-\nabla w(x_t), x\rangle\}$, by optimality we have $\langle \nabla w(x_{t+1}) + \gamma_t g(x_t) - \nabla w(x_t), x-x_{t+1}\rangle \ge 0$. Therefore, we have $\langle \gamma_t g(x_t), x_{t+1} -x \rangle \le \langle \nabla w(x_{t+1}) - \nabla w(x_t), x-x_{t+1} \rangle = V_w(x, x_t) - V_w(x, x_{t+1}) - V_w(x_{t+1}, x_t)$. Using $\langle \gamma_t g(x_t), x_t - x_{t+1}\rangle \le \frac{\gamma_t^2}{2}\|g(x_t)\|^2 + \frac{1}{2}\|x_t - x_{t+1}\|^2$ and noticing $V_w(x_{t+1}, x_t) \ge \frac{1}{2}\|x_t - x_{t+1}\|^2$ by 1-strongly convexity, we have $\langle \gamma_t g(x_t), x_t - x^* \rangle \le V_w(x^*, x_t) - V_w(x^*, x_{t+1}) + \frac{\gamma_t^2}{2}\|g(x_t)\|^2$. The rest is the same as the proof of (6.17).
\end{enumerate}


\section{Stochastic Optimization}

The goal is to $\min_{x \in X} F(x) = \frac{1}{n}\sum_{i=1}^n f_i(x)$, or more generally, $\min_{x \in X} F(x) = \E_\xi (f(x, \xi))$. For large $n$, computing the full gradient is expensive. For unknown $P(\xi)$, the gradient is intractable.

\textbf{Stochastic gradient quality for smooth functions}: (12.12) let $F(x)=\frac{1}{n}\sum_{i=1}^n f_i(x)$, where $f_i$ is $L_i$-smooth and convex, and $F$ has a global minimum $x^*$. Let $L_{\text{max}} = \max_i\{L_i\}$, then for any $x$ we have $\frac{1}{n}\sum_{i=1}^n \|\nabla f_i(x) - \nabla f_i(x^*)\|^2 \le 2L_{\text{max}}(F(x) - F(x^*))$. Proof: define $g_i(x) = f_i(x) - f_i(x^*) - \nabla f_i(x^*)^\top (x-x^*)$. Thus, $g_i(x)\ge 0$, and is convex and $L_i$-smooth. By sufficient decrease, we have $0\le g_i(x - \frac{1}{L_i}\nabla g_i(x)) \le g_i(x) - \frac{1}{2L_i}\|\nabla g_i(x)\|^2$. Thus, $g_i(x) \ge \frac{1}{2L_{\text{max}}}\|\nabla g_i(x)\|^2$. Expanding this by definition, summing over $i$, and using $\nabla F(x^*)=0$ yields the result. Note that the proof works for the general $F(x) = \E(x,\xi)$.

\textbf{Stochastic gradient descent}: do $x_{t+1} = \Pi_X (x_t - \gamma_t \nabla f(x_t, \xi_t))$. In the finite-sum problem, this is $x_{t+1} = \Pi_X(x_t - \gamma_t \nabla f_{i_t}(x_t))$, where $i_t$ is sampled uniformly at random. Thus, the \emph{gradient is unbiased}: $\E_{i_t} (\nabla f_{i_t} (x_t) \mid x_t) = \sum_i \frac{1}{n} \nabla f_i (x_t) = \nabla F(x_t)$, or $\E(\nabla f(x_t, \xi_t) \mid \xi_{[t-1]}) = \nabla F(x_t)$. The step size should diminish, i.e., $\gamma_t \rightarrow 0$, to ensure convergence, as the stochastic gradient does not necessarily equal to zero at the optima.

\textbf{Analysis}:
\begin{enumerate}
    \item \textbf{Strongly convex functions}: (12.3) assume $F(x)$ is $\mu$-strongly convex and $\E(\|\nabla f(x, \xi)\|^2) \le B^2$ for any $x \in X$. With $\gamma_t = \gamma / t$ for $\gamma > \frac{1}{2\mu}$, SGD satisfies $\E(\|x_t - x^*\|^2) \le \frac{C(\gamma)}{t}$, where $C(\gamma) = \max \{\frac{\gamma^2 B^2}{2\mu\gamma -1}, \|x_1 - x^*\|^2\}$. Proof: by projection inequality, $\|x_{t+1} - x^*\|^2 \le \|x_t - \gamma_t \nabla f(x_t, \xi_t) - x^*\|^2 = \|x_t - x^*\|^2 - 2\gamma_t \langle \nabla f(x_t, \xi_t), x_t - x^* \rangle + \gamma_t^2 \|\nabla f(x_t, \xi_t)\|^2$. Taking expectation, we have $\E(\|x_{t+1} - x^*\|^2) \le \E(\|x_t - x^*\|^2) - 2\gamma_t \E(\langle \nabla f(x_t, \xi_t), x_t - x^*\rangle) + \gamma_t^2 B^2$. Note that  $\E(\langle \nabla f(x_t, \xi_t), x_t - x^*\rangle) = \E\left[\E(\langle\nabla f(x_t, \xi_t), x_t-x^*\rangle \mid \xi_{[t-1]})\right] = \E(\langle \nabla F(x_t), x_t - x^*\rangle)$. By strong convexity, $\langle \nabla F(x_t), x_t - x^* \rangle \ge \mu \|x_t - x^*\|^2$, thus $\E(\|x_{t+1} - x^*\|^2) \le (1-2\mu \gamma_t)\E(\|x_t - x^*\|^2) + \gamma_t^2 B^2$. The result follows by induction.
    \item \textbf{Convex functions}: (12.4) let $F$ be convex and $\E(\|\nabla f(x, \xi)\|^2) \le B^2$ for any $x \in X$. SGD satisfies $\E(F(\hat{x_T}) - F(x^*)) \le \frac{R^2 + B^2 \sum_{t=1}^T \gamma_t^2}{2\sum_{t=1}^T \gamma_t}$ for $\hat{x_T} = \frac{\sum_{t=1}^T \gamma_t x_t}{\sum_{t=1}^T \gamma_t}$. Proof: use $\langle \nabla F(x_t), x_t - x^* \rangle \ge F(x_t) - F(x^*)$ in the proof of (12.3) gives $\gamma_t \E(F(x_t) - F(x^*)) \le \frac{1}{2}\E(\|x_t -x^*\|^2) - \frac{1}{2}\E(\|x_{t+1} - x^*\|^2) + \frac{1}{2}\gamma_t^2 B^2$. The result follows by recursion and convexity.
    \item \textbf{Strongly convex and smooth functions, constant step size}: (12.5) assume $F(x)$ is both $\mu$-strongly convex and $L$-smooth, and $\E(\|\nabla f(x, \xi)\|^2) \le \sigma^2 + c\|\nabla F(x)\|^2$. Then, with $\gamma_t = \gamma \le \frac{1}{Lc}$, $\E(F(x_t) - F(x^*)) \le \frac{\gamma L \sigma^2}{2\mu} + (1-\gamma \mu)^{t-1}(F(x_1) - F(x^*))$.
    \item \textbf{Non-convex but smooth function}: (12.8) assume $F(x) = \E(f(x, \xi))$ is $L$-smooth and $\E(\|\nabla f(x, \xi) - \nabla F(x)\|^2) \le \sigma^2$, then with $\gamma_t = \min\{1/L, \frac{\gamma}{\sigma\sqrt{T}}\}$, SGD achieves $\E(\|\nabla F(\hat{x}_T)\|^2) \le \frac{\sigma}{T}\left(2(F(x_1) - F(x^*))/\gamma + L\gamma\right)$, where $\hat{x}_T$ is selected uniformly at random from $\{x_1, \dots, x_T\}$. Proof: by $L$-smoothness and $x_{t+1}=x_t-\gamma_t \nabla f(x_t, \xi_t)$, we have $\E(F(x_{t+1}) - F(x_t)) \le \E(\nabla F(x_t)^\top (x_{t+1} - x_t) + \frac{L}{2}\|x_{t+1} - x_t\|^2) \le \E(-\gamma_t \nabla F(x_t)^\top \nabla f(x_t, \xi_t) + \frac{L\gamma_t^2}{2}\|\nabla f(x_t, \xi_t)\|)$. Using $\E(\nabla f(x_t, \xi_t) \mid x_t) = \nabla F(x_t)$ and $\E(\|\nabla f(x_t, \xi_t)\|^2 \mid x_t) \le \sigma^2 + \|\nabla F(x_t)\|^2$, this implies $\E(F(x_{t+1}) - F(x_t)) \le -\frac{\gamma_t}{2}\E(\|\nabla F(x_t)\|^2) + \frac{L\sigma^2\gamma_t^2}{2}$ since $\gamma_t \le 1/L$. The result follows by induction.
\end{enumerate}

\textbf{Variants of SGD}
\begin{enumerate}
    \item \textbf{AdaGrad}: do $v_t = v_{t-1} + \nabla f(x_t, \xi_t)^{\odot 2}$ and $x_{t+1} = x_t - \frac{\gamma_0}{\epsilon+\sqrt{v_t}} \odot \nabla f(x_t, \xi_t)$, where $\odot$ means element-wise operation. Idea: adjust learning rate for different coordinate, use smaller step-size for mostly updates coordinates. Problem: learning rate is adjusted too aggressively and becomes too small at a later stage.
    \item \textbf{RMSProp}: do $v_t = \beta v_{t-1} + (1-\beta)\nabla f(x_t, \xi_t)^{\odot 2}$ and $x_{t+1} = x_t - \frac{\gamma_0}{\epsilon+\sqrt{v_t}} \odot \nabla f(x_t, \xi_t)$. Idea: use a moving aqerage as the discount factor so that the discount factor does not grow too fast. $\beta$ is chosen close to 1.
    \item \textbf{Adam}: do $v_t = \beta_2 v_{t-1}+(1-\beta_2)\nabla f(x_t, \xi_t)^{\odot 2}$, $m_t = \beta_1 m_{t-1} + (1-\beta_1)\nabla f(x_t, \xi_t)$ and $x_{t+1} = x_t - \frac{\gamma_0}{\epsilon + \sqrt{v_t / (1-\beta_2^t)}}\odot \frac{m_t}{1-\beta_1^t}$. Idea: combining momentum with learning rate adjustment. $\beta_1$ and $\beta_2$ are chosen close to 1.
\end{enumerate}

\textbf{Variance Reduction Technique}: the convergence guarantee of SGD with constant step size depends on the variance of the gradient. Reducing its variance makes it converge to a nearer point to the optimal.
\begin{enumerate}
    \item \textbf{Mini-batch}: use a mini-batch to estimate gradient, i.e., $\nabla f(x_t, \xi_t) = \frac{1}{b}\sum_{i=1}^b \nabla f(x_t, \xi_{t, i})$.
    \item \textbf{Momentum}: do $x_{t+1}=x_t - \gamma_t \hat{m}_t$, where $\hat{m}_t = c\sum_{\tau=1}^t \alpha^{t-\tau} \nabla f_{i_\tau}(x_\tau)$ is the weighted average of the past stochastic gradients.
    \item \textbf{Control variate}: assume we want to estimate $\theta=\E X$, and we know a random variable $Y$ that is highly correlated with $X$ and $\E Y$ can be computed easily. Then $\hat{\theta}_\alpha := \alpha(X-Y)+\E Y$ has smaller bias and larger variance when $\alpha$ increases from 0 to 1. When $\alpha=0$ it has zero variance and when $\alpha=1$ it has zero bias.
\end{enumerate}

\textbf{Stochastic variance-reduced algorithms}:
\includegraphics[width=\linewidth]{imgs/SVRG.jpg}
\begin{enumerate}
    \item \textbf{Stochastic average gradient (SAG)}: use average of the past gradient as an estimate of the full gradient: $g_t = \frac{1}{n}\sum_{i=1}^n v_i^t$, where $v_i^t = \nabla f_{i_t}(x_t)$ for $i=i_t$ and $v_i^t=v_i^{t-1}$ otherwise. Equivalently, $g_t = g_{t-1} + \frac{1}{n}(\nabla f_{i_t}(x_t) - v_{i_t}^{t-1})$. This means SAG is as cheap as SGD. It has convergence rate linear in $\log{1/\epsilon}$ for strongly convex and smooth functions.
    \item \textbf{SAGA}: use $g_t = \nabla f_{i_t}(x_t) - v_{i_t}^{t-1} + \frac{1}{n}\sum_{i=1}^n v_i^{t-1}$ to make it unbiased. It has the same rate as SAG.
    \item \textbf{Stochastic variance-reduced gradient (SVRG)}: use a fixed reference point to estimate the gradient: $g_t = \nabla f_{i_t}(x_{t}) - \nabla f_{i_t}(\tilde{x}) + \nabla F(\tilde{x})$ and $x_{t+1} = x_{t} - \eta g_t$, where the reference point $\tilde{x}_t$ is updated only once a while. Idea: $\E(\|g_t - \nabla F(x_t)\|^2) \le \E(\|\nabla f_{i_t}(x_t) - \nabla f_{i_t}(\tilde{x})\|^2) \le L_{\text{max}}^2\|x_t - \tilde{x}\|^2$, so the variance is bounded by how close $x_t$ is to $\tilde{x}$. Typical choice of $\tilde{x}$: each epoch $s$ consists of $m$ update, then use $\tilde{x}^s = \frac{1}{m}\sum_i x_t$ as the reference point of next epoch.
\end{enumerate}

\textbf{Convergence of SVRG}: (12.11) assume $f_i(x)$ is convex and $L$-smooth and $F(x)=\frac{1}{n}\sum_{i=1}^n f_i(x)$ is $\mu$-strongly convex. Let $x^*=\argmin_x F(x)$. For large $m$ and $\eta < \frac{1}{2L}$, and $\rho:=\frac{1}{\mu\eta(1-2L\eta)m}+\frac{2L\eta}{1-2L\eta} < 1$, we have $\E(F(\tilde{x}^s) - F(x^*)) \le \rho^s(F(\tilde{x}^0) - F(x^*))$. Proof: by $\|a+b\|^2\le 2\|a\|^2+2\|b\|^2$, we have $\E(\|g_t\|^2) \le 2\E(\|\nabla f_{i_t}(x_{t}) - \nabla f_{i_t}(x^*)\|^2) + 2\E(\|\nabla f_{i_t}(\tilde{x}) - \nabla f_{i_t}(x^*) - \nabla F(\tilde{x}^*)\|^2)$. Notice $\E(\|\nabla f_{i_t}(\tilde{x}) - \nabla f_{i_t}(x^*) - \nabla F(\tilde{x}^*)\|^2) = \E(\|\nabla f_{i_t}(\tilde{x}) - \nabla f_{i_t}(x^*)\|^2)$ and use lemma 12.12, we have $\E(\|g_t\|^2) \le 4L(F(x_{t}) - F(x^*) + F(\tilde{x}) - F(x^*))$. Therefore, $\E(\|x_{t+1}-x^*\|^2) = \|x_t - x^*\|^2 - 2\eta(x_t - x^*)^\top \E(g_t) + \eta^2\E(\|g_t\|^2) \le \|x_t - x^*\|^2 -2\eta(1-2L\eta)(F(x_t) - F(x^*)) + 4L\eta^2(F(\tilde{x}) - F(x^*))$. By convexity, we have $-\sum_{i=1}^m \E(F(x_{t} - F(x^*))) \le -m \E(F(\tilde{x}) - F(x^*))$. In addition, by strong convexity, we have $\E(\|\tilde{x} - x^*\|^2) \le \frac{2}{\mu}\E(F(\tilde{x}) - F(x^*))$. Combining these three gives the desired result. Set $\eta=O(1/L)$ and $m=O(L/\mu)$ makes $\rho \in (0, 0.5)$. Thus, the convergence is linear in $O(\log(1/\epsilon))$.

Other methods: 

\includegraphics[width=\linewidth]{imgs/SGD-var.jpg}

\section{Nonconvex Functions}

\textbf{Smooth functions (no longer convex)}: $f$ is called $L$-smooth if $f(y) \le f(x)+\nabla f(x)^\top (y-x) + \frac{L}{2}\|y-x\|^2$. (9.1) If $\|\nabla^2 f(x)\|\le L$ for any $x$, then $f$ is smooth. For convex $f$ and \emph{open} $\dom(f)$, the reverse is also true.

\textbf{Gradient Descent for smooth functions}: (9.2) let $f$ has a global minimum $x^*$ and is $L$-smooth. With $\gamma=1/L$, GD yields $\frac{1}{T}\sum_t \|\nabla f(x_t)\|^2 \le \frac{2L}{T}(f(x_0) - f(x^*))$. Proof: by sufficient decrease, we have $\|\nabla f(x_t)\|^2 \le 2L(f(x_t) - f(x_{t+1}))$. Sum together gives the result.

\textbf{GD with $\gamma=1/L$ cannot overshoot}: (9.3) if $x$ is not a critical point, and $f$ is $L$-smooth over the line connecting $x$ and $x^\prime = x - \gamma \nabla f(x)$. Then with $\gamma = 1/L^\prime < 1/L$, $x^\prime$ is not a critical point. This means there is no critical point between $x$ and $x^\prime$, so no overshooting.

\textbf{Deciding whether a critical point is a local minimum is coNP-complete.}

\section{Frank-Wolfe Algorithm}

This is to solve constrained optimization, similar to PGD.

\textbf{Definitions}:
\begin{enumerate}
    \item \textbf{Linear minimization oracle (LMO)}: $\LMO_X(g) = \argmin_{z\in X} g^\top z$.
    \item \textbf{Frank-Wolfe algorithm}: do $s = \LMO_X(\nabla f(x_t))$ and $x_{t+1} = (1-\gamma_t)x_t+\gamma_t s$. Benefits: (1) the iterates are always feasible for convex domain; (2) the algorithm is projection-free if we can solve LMO; (3) the iterates have a sparse representation, defined by the combination of LMO in the previous steps; (4) it is affine-invariant, meaning that affine equivalent problems can be solved at the same cost, i.e., if $g(x)=f(Ax+b)$ and $\dom(g)=A^{-1}(\dom(f)-b)$, then minimizing $g$ is the same to minimizing $f$.
    \item \textbf{Curvature constant}: the curvature constant of the constrained optimization is defined to be $C_{f,X} = \sup_{y=(1-\gamma)x+\gamma s, \gamma \in (0,1]} \frac{1}{\gamma^2} (f(y) - f(x) - \nabla f(x)^\top (y-x))$. 
\end{enumerate}

\textbf{LASSO ($L_1$ domain)}: minimize $\|Ax-b\|^2$ subject to $\|x\|_1\le 1$. Thus, $\LMO_X(g) = \argmin_{z = \pm e_i, i\in [n]} g^\top z = -\sgn(g_i) e_i$, where $i = \argmax |g_i|$. Therefore, we can compute LMO in $O(\log(d))$.

\includegraphics[width=\linewidth]{imgs/LMO.jpg}

\textbf{Duality gap as a certificate for optimization quality}: We define the duality gap at $x$ to be $g(x) = \nabla f(x)^\top (x-s)$, where $s = \LMO_X(\nabla f(x))$. (8.2) For convex $f$, $g(x) \ge f(x) - f(x^*)$. Proof: $g(x) = \nabla f(x)^\top x - \min_{z \in X} \nabla f(x)^\top z \ge \nabla f(x)^\top x - \nabla f(x)^\top x^* \ge f(x) - f(x^*)$.

\textbf{Analysis}:
\begin{enumerate}
    \item \textbf{Decrease property}: (8.4) for $\gamma_t \in [0,1]$, we have $f(x_{t+1}) \le f(x_t) - \gamma_t g(x_t) + \gamma_t^2 \frac{L}{2}\|s-x_t\|^2$, where $g(x_t)$ is the duality gap. Proof: by smoothness, $f(x_{t+1}) \le f(x_t) + \nabla f(x_t)^\top \gamma_t(s-x_t) + \gamma_t^2 \frac{L}{2}\|s-x_t\|^2 = f(x_t) - \gamma_t g(x_t) + \gamma_t^2 \frac{L}{2}\|s-x_t\|^2$.
    \item \textbf{Affine-invariant decrease}: we have $f(x_{t+1}) \le f(x_t) - \gamma_t g(x_t) + \gamma_t^2 C_{f,X}$. Proof: plug in $x=x_t$, $y=(1-\gamma_t)x_t+\gamma_t s$ into the definition of curvature constant.
    \item $\gamma_t = \frac{2}{t+2}$: (8.3) assume $f$ is convex and $L$-smooth, and $X$ is convex, closed and bounded, then $f(x_T) - f(x^*) \le \frac{2LR^2}{T+1}$, where $R = \max_{x,y\in X}\|x-y\|$. Proof: define $h(x) = f(x) - f(x^*)$, thus $h(x_{t+1}) \le h(x_t) - \gamma_t h(x_t) + \gamma_t^2 \frac{L}{2}\|s-x_t\|^2 \le (1-\gamma_t)h(x_t) + \gamma_t^2 C$, where $C = \frac{L}{2}R^2$. By induction, we have $h(x_t) \le \frac{4C}{t+1}$. (8.5) The same holds for $C_{f,X}$ instead of $C$.
    \item $\gamma_t^\prime = \argmin_{\gamma \in [0,1]} f((1-\gamma)x_t + \gamma s)$: the same result holds as now we have $h(x_{t+1}) \le h((1-\gamma_t)x_t + \gamma_t s) \le (1-\gamma_t)h(x_t) + \gamma_t^2 C$.
    \item $\gamma_t^\prime = \min(\frac{g(x_t)}{L\|s-x_t\|^2}, 1)$: the same result holds as now we have $h(x_{t+1}) \le h(x_t) - \gamma_t g(x_t) + \gamma_t^2 C$ as well, since $\gamma_t^\prime$ minimizes this upper bound.
    \item \textbf{Convergence of duality gap}: (8.7) under the same setting, $g(x_t) \le O(\frac{C_{f,X}}{T+1})$.
\end{enumerate}

\section{Newton's Method}

\textbf{Find zeros}: $x_{t+1} = x_t - \frac{f(x_t)}{f^\prime(x_t)}$. This is to solve the first-order approximation: $f(x_t)+f^\prime(x_t)(x-x_t) = 0$.

\textbf{Find minimum}: $x_{t+1} = x_t - \frac{f^\prime(x_t)}{f^{\prime\prime}(x_t)}$. This is to search zeros of $f^\prime$. Generally, $x_{t+1} = x_t - \nabla^2 f(x_t)^{-1} \nabla f(x_t)$. (10.3) $x_{t+1} = \argmin f(x_t) + \nabla f(x_t)^\top (x-x_t) + \frac{1}{2}(x-x_t)^\top \nabla^2 f(x_t) (x-x_t)$.

\textbf{Analysis}:
\begin{enumerate}
    \item \textbf{Nondegenerate quadratic function}: (10.1) for $f(x) = \frac{1}{2}x^\top M x - q^\top x +c$ with invertible symmetric $M$, Newton's method yields $x_1 = x^* = M^{-1} q$ for any $x_0$.
    \item \textbf{Affine invariance}: (10.2) let $f$ be twice differentiable, $A$ invertible and $g(y)=Ay+b$. Define $N_h(x) = x - \nabla^2 h(x)^{-1}\nabla h(x)$, then $N_{f\circ g}=g^{-1} \circ N_f \circ g$.
    \begin{center}
        \includegraphics[width=.4\linewidth]{imgs/newton-affine.jpg}
    \end{center}
    \item \textbf{Bounded inverse Hessian and Lipschitz continuous Hessian}: (10.4) assume $f$ is twice continuously differentiable, $\|\nabla^2 f(x)^{-1}\| \le 1/\mu$ and $\|\nabla^{2} f(x) - \nabla^2 f(y)\| \le B\|x-y\|$ for any $x, y \in X$. Then we have $\|x_{t+1}-x^*\| \le \frac{B}{2\mu}\|x_t - x^*\|^2$ if a critical point $x^*$ exists in $X$. Proof: Define $H(x)=\nabla^2 f(x)$. $x_{t+1}-x^* = x_t - x^* + H(x_t)^{-1}(\nabla f(x^*) - \nabla f(x_t)) = x_t - x^* + H(x_t)^{-1} \int_0^1 H(x_t+u(x^*-x_t))(x^* - x_t) du = H(x_t)^{-1} \int_0^1 (H(x_t + u(x^* - x_t)) - H(x_t))(x^* - x_t) du$. Therefore, we have $\|x_{t+1}-x^*\| \le \|H(x_t)^{-1}\| \|x^* - x_t\| \int_0^1 \|H(x_t + u(x^* - x_t)) - H(x_t)\| du \le \frac{1}{\mu} \|x_t - x^*\|^2 B \int_0^1 u du = \frac{B}{2\mu} \|x_t-x^*\|^2$.
    \item \textbf{Fast convergence if near}: (10.5) under the assumption of (10.4), if $\|x_0-x^*\| \le \mu/B$, then $\|x_T - x^*\| \le \frac{\mu}{B}(\frac{1}{2})^{2^T-1}$. Proof: induction on (10.4).
    \item \textbf{Global convergence for strongly convex and smooth functions}: with $\gamma=\mu/L$, (scaled) Newton's method $x_{t+1}=x_t - \gamma \nabla^2 f(x_t)^{-1} \nabla f(x_t)$ satisfies $f(x_t) - f^* \le (1-\frac{\mu^2}{L^2})^t (f(x_0) - f^*)$. Note that the constant is worse than GD. Proof: expand $f(x_{t+1}) - f(x_t)$ by smoothness, then use $\frac{1}{L}\le \|H_t^{-1}\| \le \frac{1}{\mu}$ and $\|\nabla f(x_t)\|^2 \ge 2\mu (f(x_t)-f^*)$.
\end{enumerate}

\section{Quasi-Newton Methods}

Computing and inverting the Hessian in Newton's method is costly. Quasi-Newton methods want to avoid this.

\textbf{Secant condition}: use approximation $f^\prime(x_t) \approx \frac{f(x_t) - f(x_{t-1})}{x_t - x_{t-1}}$ to find zeros. Thus, the second method updates as: $x_{t+1} = x_t - f^\prime(x_t) \frac{x_t - x_{t-1}}{f^\prime(x_t) - f^\prime(x_{t-1})}$. To generalize to higher dimensions, we want to find $H_t$ such that $\nabla f(x_t) - \nabla f(x_{t-1}) = H_t(x_t - x_{t-1})$, and do $x_{t+1} = x_t - H_t^{-1} \nabla f(x_t)$.

\textbf{Quasi-Newton method}: if the method satisfies secant condition, we say it is a quasi-Newton method. In the multidimensional case, $H_t$ is not unique. In particular, Newton's method is a quasi-Newton method iff $f$ is a non-degenerate quadratic function. For efficiency, quasi-Newton methods typically deal with the inverse directly.

\textbf{Greenstadt's approach}: update $H_t^{-1} = H_{t-1}^{-1} + E_t$ for some symmetric $E_t$ with small $\|E_t\|_F^2 = \sum_{i,j} e_{i,j}^2$. To introduce more flexibility, we minimize $\|AE_t A^\top\|_F^2$ for some fixed invertible matrix $A$. The program is: minimize $\frac{1}{2}\|AE_t A^\top\|_F^2$ such that $E_t^\top = E_t$ and $E_t (\nabla f(x_t) - \nabla f(x_{t-1})) = x_t - x_{t-1} - H_{t-1}^{-1} (\nabla f(x_t) - \nabla f(x_{t-1}))$, which is the secant condition. Note that this is a convex program, so we solve by Lagrange method.

For simplicity, we write the program as: minimize $\frac{1}{2}\|AE A^\top\|_F^2$ such that $Ey=r$ and $E^\top - E=0$. Define $f(E) = \frac{1}{2}\|AE A^\top\|_F^2$, thus $\nabla f(E) = A^\top A E A^\top A$. Note that the constraints are in fact linear over $e_{i,j}$. (11.3) Define $M = (A^\top A)^{-1}$ which is positive definite, then a solution $E^*$ is optimal iff $E^*  = M(\lambda y^\top + \Gamma^\top - \Gamma)M$, where $\lambda \in \mathbb{R}^{d\times 1}$ and $\Gamma \in \mathbb{R}^{d\times d}$. Solving the linear system gives $E^* = \frac{1}{y^\top M y}(r y^\top M + M y r^\top - \frac{y^\top r}{y^\top M y} Myy^\top M)$. This is called the Greenstadt method with parameter $M$.

\textbf{BFGS}: BFGS (named after four people) is the Greenstadt method with parameter $H_t^{-1}$. Note that $H_t^{-1}$ is not yet known in the computation of $E_t$, but we have $H_t^{-1} y = x_t - x_{t-1}$ and in the formula of $E^*$, $M$ appears in the form of $My$. This allows us to compute $E^*$ without knowing the value of $M$. BFGS ensures that if $f$ is not flat between $x_{t-1}$ and $x_t$, and $H_{t-1}$ is positive definite, then $H_{t}$ is also positive definite. This reduces per iteration cost to $O(d^2)$.

\textbf{L-BFGS}: Define $\sigma = x_t - x_{t-1}$, $y = \nabla f(x_t) - \nabla f(x_{t-1})$, $H = H_{t-1}^{-1}$ and $H^\prime = H_t^{-1}$, then BFGS can be written as $ H^\prime = (I-\frac{\sigma y^\top}{y^\top \sigma}) H (I-\frac{ y \sigma^\top}{y^\top \sigma}) + \frac{\sigma \sigma^\top}{y^\top \sigma}$. L-BFGS relies on the efficient computation of $H^\prime g^\prime$ given that $H g$ can be computed efficiently for any $g$ and $g^\prime$. This is because $H^\prime g^\prime = (I-\frac{\sigma y^\top}{y^\top \sigma}) \left[H (I-\frac{ y \sigma^\top}{y^\top \sigma}) g^\prime\right] + \sigma \frac{\sigma^\top g^\prime}{y^\top \sigma}$ and note $(I-\frac{\sigma y^\top}{y^\top \sigma}) s = s - \sigma \frac{y^\top s}{y^\top \sigma}$, thus can be computed in $O(d)$ with one call of $Hg$. Recursively, we get $O(td)$ time complexity, which is not helpful. Instead, L-BFGS only does recursion for $m$ times, and use $H_0$ instead of $H_{t-m}$ (which is unknown to us as we do not explicitly compute this) as the result of $m$-th recursion. Note that we start close, so $H_0$ is close to $H_{t-m}$. In practice, we use a better choice than $H_0$.  The final complexity is $O(md)$ per iteration.

\section{Modern Second-order methods and nonconvex optimization}

\subsection{Cubic Regularization}

Motivation: for $L$-Lipschitz Hessian as in the assumption of Newton's method, we have $f(x) \le f(x_t) + \nabla f(x_t)^\top (x-x_t) + \frac{1}{2}(x-x_t)^\top \nabla^2 f(x_t) (x- x_t) + \frac{L}{6}\|x-x_t\|^3$. Similar to GD which minimizes the upper bound given by smoothness (Lipschitz gradient), cubic regularization is to minimize the upper bound given by Lipschitz Hessian. Algorithm: $x_{t+1} = \argmin_x f(x_t) + \nabla f(x_t)^\top (x-x_t) + \frac{1}{2}(x-x_t)^\top \nabla^2 f(x_t) (x- x_t) + \frac{L}{6}\|x-x_t\|^3$. This can be reduced to a convex problem.

\textbf{Convergence rate}: $\min_{i\le t} \|\nabla f(x_i)\| = O(t^{-2/3})$. If convex, we have $f(x_t) - f^* = O(t^{-2})$.

\includegraphics[width=\linewidth]{imgs/cubic.jpg}

\subsection{Nonconvex Optimization}

A nonconvex optimization may have exponentially many local minima, and determining whether a critical point is a local minimum is co-NP complete.

\textbf{Classification of stationary points}: (1) If $\nabla^2 F(x) \succ 0$, then $x$ is a local minimum; (2) If $\nabla^2 F(x) \prec 0$, then $x$ is a local maximum; (3) If $\nabla^2 F(x)$ has both positive and negative eigenvalues, then $x$ is a (strict) saddle point; (4) Otherwise, it remains inconclusive.

\textbf{Analysis}:
\begin{enumerate}
    \item \textbf{SGD converges to a stationary point}: see (12.8).
    \item \textbf{SGD with random initialization}: with random initialization, GD converges to a local minimum almost surely.
    \item \textbf{Noisy SGD}: with extra noise added to the SGD update, if $f$ satisfies strict saddle property and has Lipschitz Hessian, then noisy SGD converges to a second-order stationary point.
    \item \textbf{Benign landscapes}: if the function satisfies PL condition, or all local minimum is global minimum, then GD (in the latter case requires random initialization) converges to the global minimum.
\end{enumerate}

\section{Smoothing Techniques}


\subsection{Convex Conjugate}
\textbf{Convex conjugate (Legendre-Fenchel)}: for any $f$, its convex conjugate is given as $f^*(y) = \sup_{x\in \dom(f)} \{x^\top y - f(x)\}$.

\textbf{Properties}:
\begin{enumerate}
    \item \textbf{Fenchel's inequality}: $f(x)+f^*(y) \ge x^\top y$ for any $x, y$. Proof by definition.
    \item \textbf{Conjugate of a conjugate}: (7.2) If $f$ is convex, lower semi-continuous and proper, then $(f^*)^* = f$. A proper convex function means $f(x)>-\infty$.
    \item \textbf{Strong convexity implies nice conjugate}: (7.3) If $f$ is $\mu$-strongly convex, then $f^*$ is continuously differentiable and $\frac{1}{\mu}$ smooth.
    \item \textbf{Conjugate of function sum}: If $f$ and $g$ is lower semi-continuous and convex, then $(f+g)^*(x) = \inf_y\{f^*(y)+g^*(x-y)\}$.
\end{enumerate}

\subsection{Nesterov's Smoothing}
Use $f_\mu(x) = \max_{y \in \dom(f^*)} \{x^\top y - f^*(y) - \mu d(y)\}$ as the surrogate function to minimize. $d(y)$ is a $1$-strongly convex and nonnegative everywhere function, called the proximity function. Typical choices include $d(y) = \frac{1}{2}\|y-y_0\|^2$ and $d(y) = \frac{1}{2}\sum w_i(y_i - y_{0,i})^2$ for $w_i\ge 1$. This makes $f_\mu(x)$ to be $\frac{1}{\mu}$-smooth, as $f^*(y)+\mu d(y)$ is $\mu$-strongly convex.

\textbf{Approximation Error}: for convex $f$ with bounded $\dom(f^*)$, we have $f(x)-\mu D^2 \le f_\mu(x) \le f(x)$, where $D^2 = \max_{y\in\dom(f^*)} d(y)$. The tradeoff between approximation error and optimization error: $f(x) - f^* \le [f(x) - f_\mu(x)] + [f_\mu(x) - \min_x f_\mu(x)]$. With AGD, we get $f(x_t)- f^* = O(\mu D^2 + \frac{R^2}{\mu t^2})$. To achieve approximation error $\epsilon$, we need $\mu = O(\frac{\epsilon}{D^2})$. Therefore, $T_\epsilon = O(\frac{RD}{\epsilon})$.

\subsection{Moreau-Yosida Smoothing}

Use $f_{\mu}(x) = \min_y \{f(y) + \frac{1}{2\mu}\|x-y\|^2\}$. This is actually the special case of Nesterov's smoothing with $d(y)=\frac{1}{2}\|y\|^2$. Proof: $f_\mu^{Nes}(x) = \max_y\{x^\top y - f^*(y) - \frac{\mu}{2}\|y\|^2\}= (f^* + \frac{\mu}{2}\|\cdot\|^2)^*(x) = \inf_y \{f(y) + \frac{1}{2\mu}\|x-y\|^2\}$.

\textbf{Properties}: (1) $f_\mu(x)$ is $\frac{1}{\mu}$-smooth; (2) $\min_x f(x) = \min_x f_\mu(x)$. (3) GD reduces to proximal minimization: define $\prox_{\mu\cdot f} = \argmin_y \{f(y) + \frac{1}{2\mu}\|x-y\|^2\}$, then $x_{t+1} = x_t - \mu \nabla f_\mu(x_t) \Leftrightarrow x_{t+1} = \prox_{\mu\cdot f}(x_t)$.

\textbf{Proximal operator}: the proximal operator of convex function $f$ at $x$ is defined as $\prox_f(x) = \argmin_y\{f(y)+\frac{1}{2}\|x-y\|^2\}$. For continuous convex function $f$, $\prox_f(x)$ exists and is unique.

\textbf{Properties}:
\begin{enumerate}
    \item \textbf{Fixed point}: If $f$ is convex, then $x^* \in \argmin f(x) \Leftrightarrow x^* = \prox_f(x^*)$. Proof by definition.
    \item \textbf{Non-expansive}: $\|\prox_f(x) - \prox_f(y)\| \le \|x-y\|$. Proof: Let $\mu_x = \prox_f(x)$ and $\mu_y = \prox_f(y)$. By optimality condition, $x - \mu_x \in \partial f(\mu_x)$ and $y - \mu_y \in \partial f(\mu_y)$. By the monotonicity of gradient, $(x-\mu_x - (y-\mu_y))^\top (\mu_x - \mu_y) \ge 0$. This means $\|\mu_x - \mu_y\|^2 \le (x-y)^\top (\mu_x - \mu_y) \le \|x-y\|\|\mu_x - \mu_y\|$.
    \item \textbf{Moreau decomposition}: For any $x$, $x = \prox_f(x) + \prox_{f^*}(x)$. Proof: use $u_x \in \partial f^*(x-u_x)$.
\end{enumerate}

\textbf{Proximal Point Algorithm}: $x_{t+1} = \prox_{\gamma_t\cdot f}(x_t)$. (7.14) If $f$ is convex, then $f(x_{T}) - f^* \le \frac{\|x_0 - x^*\|^2}{2\sum_{t=0}^{T-1} \gamma_t}$. Proof: by definition, $f(x_{t+1}) + \frac{1}{2\gamma_t} \|x_{t+1} - x_t\|^2 \le f(x_t)$, which implies $f(x_{t+1}) - f(x_t) \le -\frac{1}{2\gamma_t} \|x_{t+1}-x_t\|^2$. By optimality condition, $0 \in \partial f(x_{t+1}) + \frac{1}{\gamma_t}(x_{t+1} - x_t)$, which implies $\frac{x_t - x_{t+1}}{\gamma_t} \in \partial f(x_{t+1})$. Therefore, by the definition of subgradient, $f(x_{t+1}) - f^* \le \frac{1}{\gamma_t} (x_t - x_{t+1})^\top (x_{t+1} - x^*) \le \frac{1}{\gamma_t} [(x_t - x^*)^\top (x_{t+1} - x^*) - \|x_{t+1} - x^*\|^2]$. Using $(x_t - x^*)^\top (x_{t+1} - x^*) \le \frac{1}{2}(\|x_t - x^*\|^2 + \|x_{t+1} - x^*\|^2)$, we get $ f(x_{t+1}) - f^* \le \frac{1}{2\gamma_t}[\|x_t - x^*\|^2 - \|x_{t+1} - x^*\|^2]$. The rest follows from non-increasing $f(x_t)$ and summing this inequality by $t$.

\subsection{Lasry-Lions Smoothing}

Use $f_{\mu,\delta}(x) = \max_y \min_z \{f(z) + \frac{1}{2\mu}\|z-y\|^2 - \frac{1}{2\delta} \|y-x\|^2\}$. This is double application of Moreau smoothing with function flipping. If $f$ is 1-Lipschitz, then choose $\delta,\mu = O(\epsilon)$ guarantees $\epsilon$ approximation error, and $f_{\mu,\delta}$ is $O(1/\epsilon)$-smooth. This can be applied to nonconvex functions, but has computation inefficiency in this case. 

\subsection{Randomized Smoothing}

Use $f_\mu(x) = \E_z[f(x + \mu z)]$, where $z$ is an isotopic Gaussian or uniform random variable. Choosing $\mu = O(\epsilon)$ guarantees $\epsilon$ approximation error, and $f_\mu(x)$ is $O(\frac{\sqrt{d}}{\epsilon})$-smooth.

\section{Min-Max Optimization}

The min-max problem is defined as $\min_{x\in X} \max_{y \in Y} \phi(x,y)$.

\textbf{Definitions}:
\begin{enumerate}
    \item \textbf{Saddle point and minimax point}: $(x^*, y^*)$ is a saddle point if $\phi(x^*, y) \le \phi(x^*, y^*) \le \phi(x, y^*)$ for any $x, y$. $(x^*, y^*)$ is a global minimax point if $\phi(x^*, y) \le \phi(x^*, y^*) \le \max_{y^\prime \in Y} \phi(x, y^\prime)$ for any $x, y$. Game theory interpretation: saddle point means Nash equilibrium, where no play has the incentive to make unilateral changes; global minimax point is Stackelberg equilibrium, where it is the best response to the best response.
    \item \textbf{Convex-concave function}: $\phi(x, y)$ is convex-concave if $\phi(x, y)$ is convex for every fixed $y$ and $\phi(x,y)$ is concave for every fixed $x$.
    \item \textbf{Strongly-convex-strongly-concave function}: $\phi(x, y)$ is strongly-convex-strongly-concave if $\phi(x, y)$ is $\mu_1$-strongly convex for every fixed $y$ and $\phi(x,y)$ is $\mu_2$-strongly concave for every fixed $x$.
    \item \textbf{Smoothness jointly in $x$ and $y$}: $\|\nabla_x \phi(x_1, y_1) - \nabla_x \phi(x_2, y_2)\| \le L(\|x_1 - x_2\| + \|y_1 - y_2\|)$ and $\|\nabla_y \phi(x_1, y_1) - \nabla_y \phi(x_2, y_2)\| \le L(\|x_1 - x_2\| + \|y_1 - y_2\|)$.
    \item \textbf{Duality gap}: defined to be $\max_y \phi(\hat{x}, y) - \min_x \phi(x, \hat{y})$. When duality gap is smaller than or equal to $\epsilon$, we say $(\hat{x}, \hat{y})$ is an $\epsilon$-saddle point.
    \item \textbf{Monotone operators}: A operator $F$ is monotone if $(F(u) - F(v))^\top (u-v) \ge 0$ for any $u,v$; it is $\mu$-strongly monotone if $(F(u) - F(v))^\top (u-v) \ge \mu \|u-v\|^2$ for any $u,v$.
\end{enumerate}

\textbf{Properties}:
\begin{enumerate}
    \item \textbf{Characterization of saddle points}: Define $\bar{\phi}(x) = \max_y \phi(x,y)$ and $\underline{\phi}(y) = \min_x \phi(x,y)$. Then $(x^*, y^*)$ is a saddle point iff $\max_y \min_x \phi(x,y) = \min_x \max_y \phi(x,y)$, $x^* \in \argmin_x \bar{\phi(x)}$ and $y^* \in \argmax_y \underline{\phi}(y)$. Proof: by definition.
    \item \textbf{Minimax Theorem}: If $X$ and $Y$ are closed convex sets, one of them is bounded, and $\phi(x,y)$ is a continuous convex-concave function, then there exists a saddle point on $X \times Y$ and $\max_y \min_x \phi(x, y) = \min \max \phi(x, y)$.
\end{enumerate}

\textbf{Gradient Descent Ascent}: do $x_{t+1} = \Pi_X(x_t - \gamma \nabla_x \phi(x_t, y_t))$ and $y_{t+1} = \Pi_Y (y_t + \gamma \nabla_y \phi(x_t, y_t))$. This simultaneously updates $x$ and $y$.

\textbf{Analysis}:
\begin{enumerate}
    \item \textbf{SC-SC and smooth fucntions}: (12.5) with $\gamma = \frac{\mu}{4L^2}$, GDA converges linearly: $\|x_{t+1} - x^*\|^2 + \|y_{t+1} - y^*\|^2 \le (1-4\mu^2/L^2)(\|x_t - x^*\|^2 + \|y_t - y^*\|^2)$. Proof: by SC-SC, $(\nabla_x f(x,y) - \nabla_x f(x^*, y^*))^\top (x - x^*) + (\nabla_y f(x^*, y^*) - \nabla_y f(x,y))^\top (y-y^*) \ge \mu\|x - x^*\|^2 + \mu\|y-y^*\|^2$. By smoothness, $\|\nabla_x \phi(x_1, y_1) - \nabla_x \phi(x_2, y_2)\|^2 + \|\nabla_y \phi(x_1, y_1) - \nabla_y \phi(x_2, y_2)\|^2 \le 4L(\|x_1 - x_2\|^2 + \|y_1 - y_2\|^2)$. Using these in the update gives the result.
    \item \textbf{C-C functions}: may not converge, e.g. $\phi(x,y) = xy$. Proof: $x_{t+1}^2 + y_{t+1}^2 = (x_t - \gamma y_t)^2 + (y_t + \gamma x_t)^2 = (1+\gamma^2)(x_t^2 + y_t^2)$.
\end{enumerate}

\textbf{Extragradient Method}: 

\includegraphics[width=\linewidth]{imgs/EG.jpg}

This is different from two steps of GDA, because the second step still uses the value of $t$. Essentially, EG is to use better gradients than GDA. EG has $O(1/T)$ convergece rate of the duality gap for averaged $x_{t+1/2}$ and $y_{t+1/2}$ in the C-C setting with bounded domain, and $O(\log(\frac{1}{\epsilon}))$ complexity in the SC-SC setting, which is optimal.

\textbf{Optimistic GDA}

\includegraphics[width=\linewidth]{imgs/OGDA.jpg}

This is equivalent to:

\includegraphics[width=\linewidth]{imgs/OGDA2.jpg}

OGDA has similar convergence rate as EG in the SC-SC and C-C settings.

\textbf{Proximal Point Algorithm}

do $(x_{t+1}, y_{t+1}) = \argmin_x \argmax_y \{\phi(x,y) + \frac{1}{2\eta} \|x-x_t\|^2 - \frac{1}{2\eta} \|y - y_t\|^2\}$.

This is equivalent to

\includegraphics[width=\linewidth]{imgs/PPA.jpg}

PPA converges with $O(1/T)$ in the C-C setting. PPA, EG, OGDA and GDA are all using some estimation of the gradient.

\textbf{Concave games}: finite number of players $\mathcal{N} = \{1,\dots,N\}$, compact convex action set $X_i$, the utility function $u_i(x_i, x_{-i})$ is concave in $x_i$ for all $x_{-i}$.

\textbf{Nash equilibrium for concave games}: $u_i(x^*_i, x^*_{-i}) \ge u_i(x_i, x^*_{-i})$ for any $x_i \in X_i$. First-order characterization: in the concave game, Nash equilibria can be characterized as $\nabla_i u_i(x^*_i, x^*_{-i})^\top (x_i-x^*_i) \le 0$.

\textbf{Variational inequality problem}: find $z^* \in Z$ such that $F(z^*)^\top (z-z^*) \ge 0$ for all $z \in Z$. Existence: If $Z$ is a nonempty convex compact set and $F$ is continuous, then VI has a solution.

\textbf{Strong and weak solution of VI}: $z^*$ is a strong solution if $F(z^*)^\top (z-z^*) \ge 0$ for any $z$; $z^*$ is a weak solution if $F(z)^\top (z-z^*) \ge 0$. If $F$ is monotone, then a strong solution is a weak solution; If $F$ is continuous, then a weak solution is a strong solution.

\textbf{Min-max is a special case of VI}: let $F = (\nabla_x \phi, -\nabla_y \phi(y))$ makes it a VI. A point is a solution to this VI iff it is a saddle point.

\textbf{Solving VI by EG}: assume $Z$ is a closed convex set, the VI problem has solutions, $F$ is monotone and $L$-Lipschitz continuous, then EG: $\tilde{z_{t+1}} = \Pi_Z(z_t - \eta_t F(z_t))$ and $z_{t+1} = \Pi_Z(z_t - \eta_t F(\tilde(z)_{t+1}))$ converges in $O(1/T)$.

\end{multicols*}
\end{document}
